\section{Outer measure and Carathéodory construction}

Constructing a measure with desired properties is difficult, but it is much easier to construct {\em outer measure}, since it has less restrictive properties. Then the Carathéodory theorem shows how to construct a measure from an outer measure.

\medskip

\begin{definition}
Let $X$ be a set. A function $\mu^*: P(X) \to [0,\infty]$ is called outer measure if 
\begin{enumerate}[label=(\alph*)]
    \item $\mu^*(\emptyset) = 0$;
    
    \item If $A \subset B$, the $\mu^*(A) \leq \mu^*(B)$;
    
    \item For all sets $A_1, A_2, \cdots \subset X$,
    \begin{align*}
        \mu^* \left( \bigcup^\infty_{i=1} A_i \right) \leq \sum^\infty_{i=1} \mu^*(A_i).
    \end{align*}
\end{enumerate}
\end{definition}

\medskip

\begin{definition}
A set $E$ is said to be $\mu^*$-measurable if for all $A \subset X$,
\begin{align}\label{equation_14}
    \mu^*(A) = \mu^*(A \cup E) + \mu^*(A \setminus E).
\end{align}
Condition (\ref{equation_14}) is called Carathéodory condition.
\end{definition}

\begin{remark}
The inequality 
\begin{align*}
    \mu^*(A) \leq \mu^*(A \cup E) + \mu^*(A \setminus E)
\end{align*}
always holds. In order to verify measurability of a set $E$, it suffices to show that 
\begin{align*}
    \mu^*(A) \geq \mu^*(A \cup E) + \mu^*(A \setminus E).
\end{align*}
\end{remark}

\medskip

\begin{proposition}\label{prop_14}
All sets with $\mu^*(E) = 0$ are $\mu^*$-measurable.
\end{proposition}
\begin{proof}
If $\mu^*(E) = 0$, then for an arbitrary set $A \subset X$, we have
\begin{align*}
    \mu^*(A) \geq \mu^*(A \setminus E) = \mu^*(A \setminus E) + \underbrace{\mu^*(A \cap E)}_{0},
\end{align*}
where the last term follows from $(A \cap E) \subset E$.
\end{proof}

\medskip

\begin{definition}
Let $\MM^*$ be the class of all $\mu^*$-measurable sets.
\end{definition}

\medskip

\begin{theorem}[{\bf Carathéodory}]\label{caratheodory_theorem}
$\MM^*$ is a $\sigma$-algebra and $\mu^*: \MM^* \to [0, \infty]$ is a measure.
\end{theorem}
\begin{proof}
We will prove this theorem in seven steps.
\begin{enumerate}[label=(\Roman*)]
    \item $X \in \MM^*$. This is obvious. \label{Caratheodory_step1}
    
    \item If $E, F \in \MM^*$, then $E \cup F \in \MM^*$.
    
    For fixed subset $A \subset X$, we have
    \begin{align*}
        \mu^*(A) & = \mu^*(A \cap E) + \mu^*(A \setminus E) \\
        & = \mu^*(A \cap E) + \mu^*((A \setminus E) \cap F) + \mu^*(A \setminus E) \setminus F).
    \end{align*}
    Since $A \cap E = A \cap (E \cup F) \cap F$ and $(A \setminus E)\cap F = (A \cap (E \cup F)) \setminus E$, then 
    \begin{align*}
        \mu^*(A) & = \mu^*(A \cap (E \cup F)) + \mu^*((A \cap (E \cup F)) \setminus E) + \mu^*(A \setminus (E \cup F)) \\
        & = \mu^*(A \cap (E \cup F)) + \mu^*(A \setminus (E \cup F)),
    \end{align*}
    and hence $E \cup F \in \MM^*$. \label{Caratheodory_step2}
    
    \item If $E \in \MM^*$, then $X \setminus E \in \MM^*$. For any $A \subset X$, we have
    \begin{align*}
        \mu^*(A) & = \mu^*(A \cap E) + \mu^*(A \setminus E) \\
        & = \mu^*(A \setminus (X \setminus E)) + \mu^*(A \cap (X \setminus E)),
    \end{align*}
    and hence $E \in \MM^*$. \label{Caratheodory_step3}
    
    \item If $E, F \in \MM^*$, then $E \setminus F \in \MM^*$. Since $E \setminus F = X \setminus ((X \setminus E) \cup F)$, the claim follows. \label{Caratheodory_step4}
    
    \item If $E_1, E_2, \cdots \in \MM^*$ are pairwise disjoint, then for any $A \subset X$,
    \begin{align*}
        \mu^* \left(A \cap \bigcup^\infty_{i=1} E_i\right) = \sum^\infty_{i=1} \mu^*(A \cap E_i).
    \end{align*}
    Indeed, suppose $E, F \in \MM^*$ are disjoint subsets, then for any $A \subset X$,
    \begin{align*}
        \mu^*(A \cap (E \cup F)) & = \mu^*(A \cap (E \cup F) \cap E) + \mu^*(A \cap (E \cup F) \setminus E) \\
        & = \mu^*(A \cap E) + \mu^*(A \cap F).
    \end{align*}
    Step \ref{Caratheodory_step2} and induction imply that for every $n \in \mathbb{N}$,
    \begin{align*}
        \mu^* \left(A \cap \bigcup^n_{i=1} E_i\right) = \sum^n_{i=1} \mu^*(A \cap E_i).
    \end{align*}
    Then, 
    \begin{align*}
        \mu^* \left(A \cap \bigcup^\infty_{i=1} E_i\right) \geq \sum^n_{i=1} \mu^*(A \cap E_i),
    \end{align*}
    and letting $n \to \infty$ gives
    \begin{align*}
        \mu^* \left(A \cap \bigcup^\infty_{i=1} E_i\right) \geq \sum^\infty_{i=1} \mu^*(A \cap E_i).
    \end{align*}
    Since the opposite direction is obvious, then the claim follows. \label{Caratheodory_step5}
    
    \item If $E_i \in \MM^*, i = 1, 2, \cdots$ are pairwise disjoint, then $\bigcup^\infty_{i=1} E_i$ is $\mu^*$-measurable. 
    
    By Step \ref{Caratheodory_step2} and induction, we have $\bigcup^n_{i=1} E_i \in \MM^*$, and then
    \begin{align*}
        \mu^*(A) & = \mu^*\left( A \cap \bigcup^n_{i=1} E_i\right) + \mu^*\left( A \setminus \bigcup^n_{i=1} E_i\right) \\
        & \geq \sum^n_{i=1} \mu^*(A \cap E_i) + \mu^*\left( A \setminus \bigcup^n_{i=1} E_i\right),
    \end{align*}
    letting $n \to \infty$ gives
    \begin{align*}
        \mu^*(A) & \geq \sum^\infty_{i=1} \mu^*(A \cap E_i) + \mu^*\left( A \setminus \bigcup^\infty_{i=1} E_i\right) \\
        & = \mu^*\left( A \cap \bigcup^\infty_{i=1} E_i\right) + \mu^*\left( A \setminus \bigcup^\infty_{i=1} E_i\right).
    \end{align*}
    Since the opposite direction is obvious, then the claim follows. \label{Caratheodory_step6}
    
    \item By Step \ref{Caratheodory_step1}, \ref{Caratheodory_step4}, \ref{Caratheodory_step6} and Proposition \ref{prop_11}, $\MM^*$ is a $\sigma$-algebra. 
    
    Also, $\mu^*(\emptyset) = 0$ and applying Step \ref{Caratheodory_step5} by taking $A = X$ gives
    \begin{align*}
        \mu^* \left(\bigcup^\infty_{i=1} E_i\right) = \sum^\infty_{i=1} \mu^*(E_i),
    \end{align*}
    and thus $\mu^*|_{\MM^*}$ is a measure.
\end{enumerate}
\end{proof}

\medskip

For a measure $\mu: \MM \to [0,1]$, it may happen that $A \subset X$ is measurable, but $B \subset A$ is not. Now we introduce {\em complete measure}. 

\medskip

\begin{definition}
A measure $\mu: \MM \to [0,1]$ is said to be complete if every subset of a set of measure zero is measurable (and hence has measure zero).
\end{definition}

\medskip

It follows from Proposition \ref{prop_14} that the measure described in the Carathéodory theorem is complete.

\medskip

\begin{corollary}
The measure $\mu^*: \MM^* \to [0,\infty]$ is complete.
\end{corollary}
\begin{proof}
Suppose $A \in \MM^*$, $\mu^*(A) = 0$ and $B \subset A$, then $\mu^*(B) = 0$. By Proposition \ref{prop_14}, $B \in \MM^*$. 
\end{proof}

\medskip

Now we want to ask can it happen that $\MM^* = \{\emptyset, X\}$? The answer is that we do not know if it does not happen. Fortunately, in some cases, we can prove that $\MM^*$ is large.

Let $(X,d)$ be a metric space, and for $E, F \subset X$, we can define
\begin{align*}
    \operatorname{dist}(E,F) = \inf_{x\in E,y\in F} d(x,y),\quad \operatorname{diam}(E) = \sup_{x,y\in E} d(x,y).
\end{align*}

\medskip

\begin{definition}
An outer measure $\mu^*$ defined on subsets of a metric space $(X,d)$ is called metric outer measure if 
\begin{align*}
    \mu^*(E \cup F) = \mu^*(E) + \mu^*(F),
\end{align*}
whenever $\operatorname{dist}(E,F) > 0$.
\end{definition}

\medskip

\begin{theorem}\label{theorem_113}
If $\mu^*$ is a metric outer measure, then all Borel sets are $\mu^*$-measurable, i.e. $\mathfrak{B}(X) \subset \MM^*$.
\end{theorem}
\begin{proof}
Recall that $E \in \MM^*$ if for any $A \subset X$, $\mu^*(A) \geq \mu^*(A \cap E) + \mu^*(A \setminus E)$. Thus it suffice to show that for open $G$ , $\mu^*(A) \geq \mu^*(A \cap G) + \mu^*(A \setminus G)$, since $\mathfrak{B}(X)$ is the smallest $\sigma$-algebra that contains all open sets in $X$.

Let $G_n = \left\{x \in G \, |\, \operatorname{dist}(x, X \setminus G) > 1/n \right\}$, and then 
\begin{align*}
    \operatorname{dist}(G_n, X \setminus G) \geq \frac{1}{n}.
\end{align*}
Also, we define 
\begin{align*}
    D_n = G_{n+1}\setminus G_n = \left\{x \in G \, \Bigg|\, \frac{1}{n+1} < \operatorname{dist}(x, X \setminus G) \leq \frac{1}{n} \right\}.
\end{align*}
Clearly, we have 
\begin{align}\label{equation_15}
    G \setminus G_n = \bigcup^\infty_{i=n} D_i,
\end{align}
and for $j \geq i + 2$, 
\begin{align*}
    \operatorname{dist}(D_j,D_i) \geq \frac{1}{i+1} - \frac{1}{j} > 0.
\end{align*}
Now we have sets $D_1, D_3, \cdots, D_{2n-1}$ that are pairwise disjoint, for any $A \subset X$ such that $\mu^*(A) < \infty$, we have
\begin{align*}
    \sum^{n-1}_{i=0}\mu^*\left(A \cap D_{2i+1}\right) = \mu^* \left(A \cap \bigcup^{n-1}_{i=0} D_{2i+1} \right) \leq \mu^*(A),
\end{align*}
since $\mu^*$ is a measure under $\MM^*$. If $\mu^*(A) = \infty$, then the theorem follows easily. Similarly, we have
\begin{align*}
    \sum^{n}_{i=1}\mu^*\left(A \cap D_{2i}\right) \leq \mu^*(A),
\end{align*}
and letting $n \to \infty$ implies 
\begin{align*}
    \sum^\infty_{i=1} \mu^* \left(A \cup D_i \right) < 2 \mu^*(A) < \infty.
\end{align*}

Now (\ref{equation_15}) implies
\begin{align*}
    \mu^*(A \cap (\underbrace{G \setminus G_n}_{\bigcup^\infty_{i=n}D_i})) \leq \sum^\infty_{i=n} \mu^*(A \cap D_i) \xrightarrow[]{n \to \infty} 0,
\end{align*}
and since $\operatorname{dist}(G_n, X \setminus G) \geq 1/n$, $\operatorname{dist}(A \cap G_n, A\setminus G) > 0$, and hence
\begin{align*}
    \mu^*(A \cap G_n) + \mu^*(A\setminus G) = \mu^*((A \cap G_n) \cup (A\setminus G)) \leq \mu^*(A).
\end{align*}
Thus,
\begin{align*}
    \mu^*(A \cap G) + \mu^*(A\setminus G) & \leq \mu^*(A \cap G_n) + \mu^*(A \cap (G \setminus G_n)) + \mu^*(A\setminus G) \\
    & \leq \mu^*(A) + \mu^*(A \cap (G \setminus G_n)) \xrightarrow[]{n \to \infty} \mu^*(A),
\end{align*}
which proves the theorem.
\end{proof}

\medskip

Outer measure can lead to following important results.

\medskip

\begin{theorem}\label{theorem_114}
Let $X$ be a metric space and $\mu$ a measure in $\mathfrak{B}(X)$. Suppose that $X$ is a union of countable many open sets of finite measure. Then, for any $E \in \mathfrak{B}(X)$, 
\begin{align*}
    \mu(E) = \inf_{\substack{U \supset E\\ U - \text{open}}} \mu(U) = \sup_{\substack{C \subset E\\ C - \text{closed}}} \mu(C).
\end{align*}
\end{theorem}


This theorem gives a fantastic result to compute measure. For example, if there is a measure on $\mathfrak{B}(\mathbb{R})$ such that $\mu((a,b)) = b - a$, then it is easy to compute $\mu(U)$, where $U \subset \mathbb{R}$ is open and hence for $E \in \mathfrak{B}(\mathbb{R})$,
\begin{align*}
    \mu(E) = \inf_{\substack{U \supset E\\ U \in \mathbb{R} - \text{open}}} \mu(U).
\end{align*}
The existence of such a measure will be proved later.

Before proving this theorem, we talk two of its corollaries. The first one shows that in a situation described by the above theorem, in order to prove that two measures are equal, it suffices to compare them on the class of open sets.

\medskip

\begin{corollary}
If $\mu$ is a measure in the Theorem \ref{theorem_114}, and $\nu$ is another measure on $\mathfrak{B}(X)$ such that 
$$\mu(U) = \nu(U),\,\, \text{for all open sets}\,\, U \subset X,$$ 
 then 
$$\mu(E) = \nu(E),\,\, \text{for all}\,\, E \subset \mathfrak{B}(X).$$
\end{corollary}
\begin{proof}
\begin{align*}
    \mu(E) = \inf_{U \supset E} \mu(U) = \inf_{U \supset E} \nu(U) = \nu(E),
\end{align*}
where the last step comes from the theorem above.
\end{proof}

\medskip

\begin{definition}
Let $X$ be a metric space and $\mu$ is a measure defined on the $\sigma$-algebra of Borel sets. We say $\mu$ is a Radon measure if $\mu(K) < \infty$ for any compact set $K$ and 
\begin{align*}
    \mu(E) = \inf_{\substack{U \supset E\\ U - \text{open}}} \mu(U) = \sup_{\substack{K \subset E\\ K - \text{compact}}} \mu(K), \,\, \text{for all}\,\, E \subset \mathfrak{B}(X).
\end{align*}
\end{definition}

\medskip

\begin{corollary}
If $X$ is a locally compact\footnote{A metric space $X$ is called {\em locally compact} if every point $x$ of $X$ has a compact neighbourhood, i.e. there exists an open set $U$ and a compact set $K$, such that $x\in U \subseteq K$. Also, there are other definitions as well, such as every point has a neighborhood whose closure is compact. For example, $\mathbb{R}^n$ is locally compact, as well as $\mathbb{R}^n \setminus \{0\}$. Note that being locally compact does not imply it is bounded and closed.}and separable metric space and $\mu$ is a measure on $\mathfrak{B}(X)$ such that $\mu(K) < \infty$ for any compact set $K$, then $X$ is a union of countable many open sets of finite measure and $\mu$ is a Radon measure. 
\end{corollary}
\begin{proof}
Since $X$ is separable, then it has a countable, dense subsets. Also, since $X$ is locally compact, then there exists countable open sets $U_i$ such that $X = \bigcup^\infty_{i=1} U_i$, and each $\overline{U_i}$ is compact. By assumption, $\mu(U_i) \leq \mu(\overline{U_i}) < \infty$. By Theorem \ref{theorem_114},
\begin{align*}
    \mu(E) = \inf_{\substack{U \supset E\\ U - \text{open}}} \mu(U) = \sup_{\substack{C \subset E\\ C - \text{closed}}} \mu(C), \,\, \text{for all}\,\, E \subset \mathfrak{B}(X).
\end{align*}

We need to show 
\begin{align*}
    \mu(E) =  \sup_{\substack{K \subset E\\ K - \text{compact}}} \mu(K), \,\, \text{for all}\,\, E \subset \mathfrak{B}(X).
\end{align*}

Let 
\begin{align*}
    C = \bigcup^\infty_{n=1} \left(C \cap \bigcup^n_{i=1} \overline{U_i} \right),
\end{align*}
and $K = C \cap \bigcup^n_{i=1} \overline{U_i}$ is compact. By Theorem \ref{theorem_111}(f), for any closed set $C \subset X$,
\begin{align*}
    \mu(C) = \sup_{K \subset C} \mu(K),
\end{align*}
then the proof is complete.
\end{proof}

\medskip

\begin{proof}[Proof of Theorem \ref{theorem_114}]
~\begin{enumerate}[label=(\alph*)]
    \item For $E \subset X$, we define
    \begin{align*}
        \mu^*(E) = \inf_{\substack{U \supset E\\ U - \text{open}}} \mu(U),
    \end{align*}
    It is easy to see that $\mu^*$ is a metric outer measure. By Carathéodory theorem \ref{caratheodory_theorem} and Theorem \ref{theorem_113}, $\mu^*$ on $\mathfrak{B}(X)$ is a measure. Then, 
    \begin{align}\label{theorem_114_1}
        \mu(U) = \mu^*(U), \,\, \text{for all open set} \,\, U \subset X,
    \end{align}
    and since $\mu^*(E) = \inf \mu(U)$, where $E \subset U$, then
    \begin{align}\label{theorem_114_2}
        \mu(E) \leq \mu^*(E), \,\, \text{for all} \,\, E \subset \mathfrak{B}(X).
    \end{align}

    $X$ can be a union of increasing sequence of open sets with finite measure, i.e. 
    \begin{align*}
        X = \bigcup^\infty_{i=1} V_i, \quad V_i \subset V_{i+1}, \quad \mu(V_i) < \infty.
    \end{align*}
    Indeed, since $X = \bigcup^\infty_{i=1} U_i$, where $U_i$ are open and $\mu(U_i) < \infty$, taking $V_n = \bigcup^n_{i=1} U_i$ will do. Now inequality (\ref{theorem_114_2}) implies that for any $E \subset \mathfrak{B}(X)$,
    \begin{align}\label{theorem_114_3}
        \mu(V_n \setminus E) \leq \mu^*(V_n \setminus E), \quad \mu(V_n \cap E) \leq \mu^*(V_n \cap E),
    \end{align}
    since $V_n \setminus E, V_n \cap E \subset \mathfrak{B}(X)$. However, inequalities (\ref{theorem_114_3}) are a contradiction to the equality (\ref{theorem_114_1}), since 
    \begin{align*}
        \mu(V_n) = \mu(V_n \setminus E) + \mu(V_n \cap E) < \mu^*(V_n \setminus E) + \mu^*(V_n \cap E) = \mu^*(V_n),
    \end{align*}
    then we could only have equalities in (\ref{theorem_114_3}). In particular, $\mu(V_n \cap E) = \mu^*(V_n \cap E)$. Letting $n \to \infty$ and by Theorem \ref{theorem_111}(f), 
    \begin{align*}
        \mu(E) \leftarrow \mu(V_n \cap E) = \mu^*(V_n \cap E) \rightarrow \mu^*(E).
    \end{align*}
    Then $\mu(E) = \mu^*(E)$ and thus
    \begin{align*}
        \mu(E) = \inf_{\substack{U \supset E\\ U - \text{open}}} \mu(U).
    \end{align*}
    
    \item It remains to show that $\mu(E) = \sup \mu(C), C \subset E$ and $C$ is closed. Since $\mu(V_n \setminus E) < \infty$ and there exists an open set $G_n$ such that
    \begin{align*}
        \left(V_n \setminus E\right) \subset G_n, \quad \mu\left(G_n \setminus (V_n \setminus E)\right) < \frac{\varepsilon}{2^n},
    \end{align*}
    since we already proved that for any $E \subset \mathfrak{B}(X)$, $\mu(E)$ can be approximated by measure of open set that contains $E$ and clearly, $V_n \setminus E \in \mathfrak{B}(X)$ and can be approximated by the measure of open sets $G_n$. 
    
    Let $G = \bigcup^\infty_{n=1} G_n$, and then $G$ is open and $C = X \setminus G \in E$ is closed. Indeed, if $x \in C$, then $x \notin G$ and hence $x \notin G_n$ for all $n = 1, 2, \cdots$. By definition of $G_n$, $x \notin (V_n \setminus E)$ for all $n = 1, 2, \cdot$, and then $x \notin X\setminus E$, which implies $x \in E$.
    
    Now, 
    \begin{align*}
        E \setminus C = E \cap \left(\bigcup^\infty_{n=1} G_n\right) = \bigcup^\infty_{n=1}(E \cap G_n) \subset \bigcup^\infty_{n=1} \left( G_n\setminus (V_n \setminus E) \right).
    \end{align*}
    Indeed, if $x \in (E \cap G_n)$, then $x \in G_n$ or $x \in E$, which implies $x \in G_n$ or $x \notin (V_n \setminus E)$, and hence $x \in G_n\setminus (V_n \setminus E)$. Since $\mu\left(G_n \setminus (V_n \setminus E)\right) < \varepsilon / 2^n$, then 
    \begin{align*}
        \mu(E \setminus C) < \sum^\infty_{n=1} \frac{\varepsilon}{2^n} = \varepsilon,
    \end{align*}
    and since $C \subset E$, we have 
    \begin{align*}
        \mu(E) = \sup_{\substack{C \subset E\\ C - \text{closed}}} \mu(C).
    \end{align*}
\end{enumerate}
\end{proof}

\medskip
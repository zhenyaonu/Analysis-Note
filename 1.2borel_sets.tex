\section{Borel sets}

\begin{definition}
Let $X$ be a metric space (or topological space). Denote by $\mathfrak{B}(X)$ the $\sigma$-algebra generated by the family of all open sets in $X$. Elements of $\mathfrak{B}(X)$ are called Borel sets and $\mathfrak{B}(X)$ is called $\sigma$-algebra of Borel sets. So,
\begin{enumerate}[label=(\alph*)]
    \item All closed sets are Borel sets;
    
    \item If $G_1, G_2, \cdots$ are open, then $\bigcap^\infty_{i=1}G_i$ is Borel, and such sets are called $G_\delta$ sets.
    
    \item If $F_1, F_2, \cdots$ are closed, then $\bigcup^\infty_{i=1}F_i$ is Borel, and such sets are called $F_\sigma$ sets.
\end{enumerate}
The notations $G_\delta$ and $F_\sigma$ is due to Hausdorff{\rm \cite{2}}.
\end{definition}

\medskip

\begin{example}
Every closed set is $G_\delta$. Indeed, suppose $F$ is closed, and let
\begin{align*}
    G_i = \left\{x\,|\, \operatorname{dist}(x,F) < \frac{1}{i}\right\},
\end{align*}
where $G_i$ are open, and then $F = \bigcap^\infty_{i=1} G_i$.

Similarly, every open set is $F_\sigma$. Suppose $G$ is open set, and let $F = X \setminus E$, which is closed. Then,
\begin{align*}
    G = X \setminus F = X \setminus \bigcap^\infty_{i=1} G_i = \bigcup^\infty_{i=1} (X \setminus G_i),
\end{align*}
where $X \setminus G_i$ are closed.
\end{example}

\medskip

\begin{proposition}
Let $A \subset X$. $A$ is $F_\sigma$ set if and only if $X \setminus A$ is $G_\delta$.
\end{proposition}
\begin{proof}
~\begin{enumerate}[label=(\alph*)]
    \item ($\Rightarrow$) By assumption, $A = \bigcup^\infty_{i=1} F_i$, where $F_i$ are closed. Then, $X \setminus A = \bigcap^\infty_{i=1} (X \setminus F_i)$, where $X \setminus F_i$ are open. This implies $X \setminus A$ is $G_\delta$.
    
    \item ($\Leftarrow$) This direction is similar.
\end{enumerate}
\end{proof}

\medskip

\begin{example}
$\mathbb{Q}$ is a $F_\delta$ set, where $\mathbb{Q}$ is the set of all rational number.
\end{example}
\begin{proof}
Since $\mathbb{Q}$ is countable, we could write it as $\mathbb{Q} = \{q_1, q_2, \cdots\}$, and it is obvious that $\mathbb{Q} = \bigcup^\infty_{i=1} \{q_i\}$, where $\{q_i\}$ is closed. Thus, $\mathbb{Q}$ is $F_\delta$.
\end{proof}

\medskip

\begin{theorem}[{\bf Baire Category}]
Let $X$ be a complete metric space, $G_i \subset X, i = 1,2,\cdots$ are open and dense in $X$. Then, $\bigcap^\infty_{i=1} G_i$ is also dense.
\end{theorem}
\begin{proof}
The proof is from Rudin's {\em Real and Complex Analysis}\cite{2}. Let $W$ be any open set in $X$, a subset is dense if and only if every nonempty open subset intersects it. Thus, to show that the intersection is dense it suffice to show that $\bigcap^\infty_{i=1} G_i$ has a point in $W$ if $W \neq \emptyset$.

Since $G_1$ is dense, then $G_1 \cap W \neq \emptyset$, then there is a point $x_1$ and $0 < r_1 < 1$ such that
\begin{align}\label{Baire_Category_1}
    \overline{B}\left(x_1,r_1\right) \subset (G_1 \cap W),
\end{align}
where $\overline{B}(x,r)$ denotes the closure of $B(x,r)$. Since each $G_i$ is dense, continuing this process gives sequences $\{x_n\} \in X$ and $\{r_n\}$ such that 
\begin{align}\label{Baire_Category_2}
    B\left(x_n,r_n\right) \subset \left( B\left(x_{n-1},r_{n-1}\right) \cap G_n \right), \quad 0 < r_n < \frac{1}{n}.
\end{align}
Then, for $n > m$, we have $x_n \in B(x_m,r_m)$, and hence $\{x_n\}$ is a Cauchy sequence. By completeness of $X$, there exists some point $x$ in $X$ such that $\lim_{n\to\infty} x_n = x$.

Since $x_n \in \overline{B}(x_m,r_m)$ for $n > m$, then $x \in \overline{B}(x_n,r_n)$ for each $n \in \mathbb{N}$ and by (\ref{Baire_Category_2}), $x \in G_i$ for $i = 1,2,\cdots$. By (\ref{Baire_Category_1}), $x \in W$, and thus $\bigcap^\infty_{i=1} G_i$ is dense. 
\end{proof}

\medskip

\begin{example}
$\mathbb{Q}$ is not a $G_\sigma$ set, i.e. $\mathbb{Q} \in F_\delta \setminus G_\sigma$.
\end{example}
\begin{proof}
Suppose $\mathbb{Q}$ is $G_\sigma$, then $\mathbb{Q} = \bigcap^\infty_{i=1} G_i$, where $G_i$ is dense in $\mathbb{R}$. Let $\widetilde{G_i} = \mathbb{R} \setminus \{q_i\}, q_i \in \mathbb{Q}$ for $i = 1,2,\cdots$, which are open and dense. Now,
\begin{align*}
    \left(\bigcap^\infty_{i=1}G_i\right) \cap \left(\bigcap^\infty_{i=1}\widetilde{G_i}\right) = \mathbb{Q} \cap (\mathbb{R} \setminus \mathbb{Q}) = \emptyset,
\end{align*}
and this is a contradiction to Baire Category theorem.
\end{proof}

\medskip

Now what are Borel sets? Recall that open, closed sets, $G_\delta, F_\sigma$, intersections of countable many $F_\sigma$'s and union of countable many $G_\delta$'s are all in Borel sets $\mathfrak{B}(X)$, but these are only a small part of Borel sets. We need well-ordering and ordinal numbers\footnote{See Appendix \ref{appendix_a}.} to introduce Borel hierarchy, and then to construct Borel sets.

\medskip


Let's introduce the {\em finite Borel hierarchy}.

\begin{definition}[{\bf Borel sets of finite order}]
Let
\begin{align*}
    \Sigma_1 & = \{G\,|\, G \subseteq X, G\,\, \text{is open}\}, \\
    \Pi_1 & = \{F\,|\, F \subseteq X, F\,\, \text{is closed}\},
\end{align*}
and then inductively define the following collection of subsets of $X$
\begin{align*}
    \Sigma_{n+1} & = \left\{\bigcup^\infty_{i=1} F_i \,|\, F_i \in \Pi_{n}\right\}, \\
    \Pi_{n+1} & = \left\{\bigcup^\infty_{i=1} G_i \,|\, G_i \in \Sigma_{n}\right\}.
\end{align*}
\end{definition}

\medskip

\begin{theorem}
~\begin{enumerate}[label=(\alph*)]
    \item If $S \subset \mathbb{R}$, then $S \subset \Sigma_{n}$ if and only if $\mathbb{R} \setminus S \in \Pi_{n}$.
    
    \item For any $n \in \mathbb{N}$, $\Sigma_{n} \subset \Sigma_{n+1}$, $\Pi_{n} \in \Pi_{n+1}$ and $\Sigma_{n} \in \Pi_{n+1}$, $\Pi_{n} \in \Sigma_{n+1}$. Also, 
    \begin{align*}
        \left(\Pi_{n} \cap \Sigma_{n}\right) \subset \left(\Pi_{n+1} \cap \Sigma_{n+1}\right).
    \end{align*}
\end{enumerate}
\end{theorem}

\medskip

\begin{definition}
Let $B \subset \mathbb{R}$ be a Borel set. $B$ has Borel rank $n \in \mathbb{N}$ if $n$ is the least integer such that $$B \subset \left(\Sigma_n \cup \Pi_n \right) \setminus \left(\Sigma_{n-1} \cap \Pi_{n-1}\right).$$
\end{definition}

\medskip

\begin{example}
Open and closed subset of $\mathbb{R}$ has Borel rank $1$, $\mathbb{Q}$ has Borel rank $2$.
\end{example}

\medskip

\begin{theorem}
There are Borel sets of infinite rank.
\end{theorem}

\medskip

We considered all Borel sets of finite rank, but how to construct a set belong to $\mathfrak{B}(X)$ but with infinite rank?

Let $S_n \subset (n,n+1)$ with Borel rank $n$, then $S_1 \cup S_2 \cup \cdots$ is also Borel but with infinite rank. Now let $\Sigma_\omega$ be the unions of Borel sets of finite rank, and $\Pi_\omega$ be the intersections of Borel sets of finite rank. And we can define $\Sigma_{\omega+1}$ and $\Pi_{\omega+1}$ accordingly. 

\medskip

\begin{theorem}
$\mathfrak{B}(\mathbb{R}) = \bigcup (\Sigma_\omega \cup \Pi_\omega)$, where $\omega$ is countable, ordinal number.
\end{theorem}

\medskip
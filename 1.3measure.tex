\section{Measure}

\begin{definition}
Let $(X, \MM)$ be a measurable space. A measure (also called positive measure) is a function $\mu: \MM \to [0,\infty]$ such that
\begin{enumerate}[label=(\alph*)]
    \item $\mu(\emptyset) = 0$;
    
    \item $\mu$ is countably additive, i.e. if $A_1, A_2, \cdots \in \MM$ are pairwise disjoint, then
    \begin{align*}
        \mu \left( \bigcup^\infty_{i=1} A_i \right) = \sum^\infty_{i=1} \mu(A_i).
    \end{align*}
\end{enumerate}
And ``ordered triples'' $(X, \MM, \mu)$ is called a measure space. If $\mu(X) < \infty$, then $\mu$ is called finite measure. If $\mu(X) = 1$, then $\mu$ is called probability or probability measure. If $X = \bigcup^\infty_{i=1} A_i$, where $A_i \in \MM$ and $\mu(A_i) < \infty$ for all $i = 1,2,\cdots$, then we say $\mu$ is $\sigma$-finite. 
\end{definition}

\medskip

\begin{example}
If $X$ is an arbitrary set, and $\mu: P(X) \to [0,\infty]$ is defined by $\mu(E) = m$ if $E$ is finite and has $m$ elements, $\mu(E) = \infty$ if $E$ is infinite, then $\mu$ is a measure and called {\em counting measure}.
\end{example}

\begin{theorem}[{\bf Elementary properties of measures}]\label{theorem_111}
Let $(X, \MM, \mu)$ be a measure space. Then
\begin{enumerate}[label=(\alph*)]
    \item If the sets $A_1, A_2, \cdots, A_n \in \MM$ are pairwise disjoint, then $\mu \left(\bigcup^n_{i=1} A_i \right) = \sum^n_{i=1} \mu(A_i)$.
    
    \item If $A, B \in \MM$, $A \subset B$, then $\mu(A) \leq \mu(B)$.
    
    \item If $A, B \in \MM$, $A \subset B$, $\mu(B) < \infty$, then $\mu(B\setminus A) = \mu(B) - \mu(A)$.
    
    \item If $A_1, A_2, \cdots \in \MM$, then 
    \begin{align*}
        \mu \left(\bigcup^\infty_{i=1} A_i \right) \leq \sum^\infty_{i=1} \mu(A_i).
    \end{align*}
    
    \item If $A_1, A_2, \cdots \in \MM$, $\mu(A_i) = 0$ for $i = 1,2,\cdots$, then $\mu \left(\bigcup^\infty_{i=1} A_i \right) = 0$.
    
    \item If $A_1, A_2, \cdots \in \MM$, $A_1 \subset A_2 \subset \cdots$, then
    \begin{align*}
        \mu \left(\bigcup^\infty_{i=1} A_i \right) = \lim_{i\to\infty} \mu(A_i).
    \end{align*}
    
    \item If $A_1, A_2, \cdots \in \MM$, $A_1 \supset A_2 \supset \cdots$ and $\mu(A_1) < \infty$, then
    \begin{align*}
        \mu \left(\bigcap^\infty_{i=1} A_i \right) = \lim_{i\to\infty} \mu(A_i).
    \end{align*}
\end{enumerate}
\end{theorem}
\begin{proof}
~\begin{enumerate}[label=(\alph*)]
    \item Considering the sequence $A_1, A_2, \cdots, A_n, \emptyset, \emptyset, \cdots$, which are pairwise disjoint, then by definition of measure, \begin{align*}
        \mu(A_1 \cup A_2 \cup \cdots \cup A_n) & = \mu(A_1) + \cdots + \mu(A_n) + \mu(\emptyset) + \cdots + \mu(\emptyset) \\
        & = \mu(A_1) + \cdots + \mu(A_n).
    \end{align*}
    
    \item Since $B = A \cup (B)$ and $A \cap (B \setminus A) = \emptyset$, then $\mu(B) = \mu(A) + \mu(B \setminus A) \geq \mu(A)$.
    
    \item By $\mu(B) = \mu(A) + \mu(B \setminus A)$ in (b). Note that we need $\mu(B) < \infty$. Otherwise, it could happen that $\mu(A) = \mu(B) = \infty$ and then $\mu(B \setminus A) = \mu(B) - \mu(A) = \infty - \infty$.
    
    \item Since
    \begin{align*}
        \bigcup^\infty_{i=1} A_i = \underbrace{A_1}_{B_1} \cup \underbrace{(A_2\setminus A_1)}_{B_2} \cup \underbrace{(A_3 \setminus (A_1\cup A_2))}_{B_3} \cup \cdots,
    \end{align*}
    and $B_1, B_2, \cdots$ are pairwise disjoint, then 
    \begin{align}\label{equaiton_11}
        \mu(A) = \sum^\infty_{i=1} \mu(B_i) \leq \sum^\infty_{i=1} \mu(A_i).
    \end{align}
    
    \item It follows from (\ref{equaiton_11}).
    
    \item Since $A_i \subset A_{i+1}$, then 
    \begin{align*}
        \bigcup^\infty_{i=1} A_i =  \underbrace{A_1}_{B_1} \cup \underbrace{(A_2\setminus A_1)}_{B_2} \cup \underbrace{(A_3 \setminus A_2)}_{B_3} \cup \underbrace{(A_4 \setminus A_3)}_{B_4} \cup \cdots,
    \end{align*}
    and $B_i$ are pairwise disjoint. Therefore,
    \begin{align*}
        \mu \left(\bigcup^\infty_{i=1} A_i \right) = \sum^\infty_{i=1} \mu(B_i) = \lim_{i\to\infty} \sum^i_{k=1} \mu(B_k) = \lim_{i\to\infty} \mu \left(\bigcup^i_{k=1} B_k \right) = \lim_{i\to\infty} \mu(A_i).
    \end{align*}
    The limit exists because $\mu(A_i) \leq \mu(A_{i+1})$.
    
    \item Apply (f) to the sets $A_1\setminus A_i$. Let $C_i = A_1 \setminus A_i$, then $C_1 \subset C_2 \subset \cdots$ and $\mu(C_i) = \mu(A_1) - \mu(A_i)$. Also, $A \setminus \left( \bigcap^\infty_{i=1} A_i\right) = \bigcup^\infty_{i=1} C_i$, then by (f), we have
    \begin{align*}
        \mu(A_1) - \mu \left(\bigcap^\infty_{i=1} A_i\right) = \mu\left(A_1 \setminus \bigcap^\infty_{i=1} A_i\right) = \lim_{i\to\infty} \mu(C_i) = \mu(A_1) - \lim_{i\to\infty} \mu(A_i),
    \end{align*}
    and this is where we used the fact that $\mu(A_1) < \infty$.
\end{enumerate}
\end{proof}

\medskip
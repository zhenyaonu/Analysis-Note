\chapter{Exercise}

\section{Measure Theory}

\begin{exercise}
Prove that $\MM$ is a $\sigma$-algebra of subsets of $X$ if and only is
\begin{enumerate}
    \item[(a)] $X\in\MM$,
    
    \item[(b)] If $A,B\in\MM$, then $A\setminus B\in \MM$,
    
    \item[(c)] If $A_1,A_2,A_3\ldots\in\MM$ are pairwise disjoint, then $\bigcup_{i=1}^\infty A_i\in\MM$.
\end{enumerate}
\end{exercise}
\begin{proof} 
~\begin{enumerate}
    \item[($\Rightarrow$)] (a) and (c) are obvious, it remains to show (b). By definition of $\sigma$-algebra, $A \cup B \cup \emptyset \cup \emptyset \cup \cdots \in \mathfrak{M}$, which implies $A \cup B \in \mathfrak{M}$. Also, by De Morgan's Laws, we have 
    \begin{align*}
        X \setminus (A \cap B) = (X \setminus A) \cup (X \setminus B),
    \end{align*}
    and since $X \setminus A, X \setminus B \in \mathfrak{M}$, $A \cap B \in \mathfrak{M}$. Now, $A \setminus B = A \cap (X \setminus B)$, then by the results above, we have $A \setminus B \in \mathfrak{M}$.
    
    \item[($\Leftarrow$)] It suffices to show that for any $A_1, A_2,\cdots \in \mathfrak{M}$, $\bigcup^\infty_{i=1}A_i \in \mathfrak{M}$. Let $B_1 = A_1$, $B_2 = A_2 \setminus A_1$, $B_3 = A_3 \setminus (B_1 \cup B_2)$ and so on. It gives a pairwise disjoint sequence of sets $B_1, B_2, \cdots \in \mathfrak{M}$. By the assumption, $\bigcup^\infty_{i=1}A_i = \bigcup^\infty_{i=1}B_i \in \mathfrak{M}$.
\end{enumerate}
\end{proof}

\medskip

\begin{exercise}
Use counting measure $\mu$ to show that it can happen that $A_1\subset A_2\subset A_3\subset\ldots$, but
$$
\mu\left(\bigcap_{i=1}^\infty A_i\right) \neq \lim_{i\to\infty}\mu(A_i).
$$
\end{exercise}
\begin{proof}
Let $A_1 = \mathbb{N}$, $A_2 = \mathbb{N} \setminus \{0\}$, $A_3 = \mathbb{N} \setminus \{0,1\}$, $A_4 = \mathbb{N} \setminus \{0,1,2\}$ and so on. Letting $i \to \infty$, $A_i$ contains infinitely many elements and then $\lim_{i\to\infty} \mu(A_0) = \infty$.

On the other hand, $\bigcap_{i=1}^\infty A_i = \emptyset$. Indeed, if $\bigcap_{i=1}^\infty A_i \neq \emptyset$, then there exists $k \in \mathbb{N}$ such that $k \in \bigcap_{i=1}^\infty A_i$. However, $k \notin A_{k+2}$, which is a contradiction. Thus, $\mu\left(\bigcap_{i=1}^\infty A_i\right) = 0$.
\end{proof}

\medskip

\begin{exercise}
Let $X$ be a metric space and let $\mu$ be a measure in $\BB(X)$. Prove that
$$
\mu^*(E) =
\inf_{\substack{U\supset E \\ U - {\rm open}}}\mu(U)
$$
defines a metric outer measure.
\end{exercise}
\begin{proof}
~\begin{enumerate}[label=(\alph*)]
    \item First, we prove that $\mu^*$ defined above is an outer measure.
    \begin{enumerate}[label=\arabic*)]
        \item Since $\emptyset \subseteq \emptyset$, $\mu^*(\emptyset) = \inf_{\substack{U\supset E \\ U - {\rm open}}}\mu(U) \leq \mu(\emptyset) = 0$.
        
        \item For any $A \subset B$, where $A, B \in \mathfrak{B}(X)$, then
        \begin{align*}
            \mu^*(B) = \inf_{\substack{U\supset B \\ U - {\rm open}}}\mu(U) = \inf (\mu(A) + \mu(U \setminus A)) \geq \mu(A) \geq \mu^*(A).
        \end{align*}
        
        \item For any $\varepsilon > 0$ and for each $A_i \subset X$, choose $U_i \subset X$ be open such that $A_i \subset U_i$ and
        \begin{align*}
            \mu(U_i) \leq \mu^*(A_i) + \frac{\varepsilon}{2^i}.
        \end{align*}
        Note that $\bigcup^\infty_{i=1} U_i$ is an open covering of $\bigcup^\infty_{i=1} A_i$, then 
        \begin{align*}
            \mu^*\left(\bigcup^\infty_{i=1} A_i\right) \leq \mu \left( \bigcup^\infty_{i=1} U_i\right) \leq \sum^\infty_{i=1} \mu(U_i) \leq \sum^\infty_{i=1} \mu^*(A_i) + \varepsilon,
        \end{align*}
        and letting $\varepsilon \to 0$ gives subadditivity
        \begin{align*}
            \mu^*\left(\bigcup^\infty_{i=1} A_i\right) \leq \sum^\infty_{i=1} \mu^*(A_i).
        \end{align*}
    \end{enumerate}
    Hence, $\mu^*$ is an outer measure.
    
    \item Second, we prove $\mu^*$ is a metric outer measure. For any $E, F \subset X$ and $\operatorname{dist}(E,F) > 0$, then there exists $\delta > 0$ such that $\operatorname{dist}(E,F) \geq \delta$. Also, for any $0 < \varepsilon < \delta$, there exist two open disjoint sets $U, V \subset X$ such that $E \subset U, F \subset V$ and 
    \begin{align*}
        \mu(U \cup V) \leq \mu^*(E \cup F) + \varepsilon.
    \end{align*}
    Then, we have
    \begin{align*}
        \mu^*(E) + \mu^*(F) \leq \mu(U) + \mu(V) = \mu(U \cup V) \leq \mu^*(E \cup F) + \varepsilon,
    \end{align*}
    and letting $\varepsilon \to 0$ gives
    \begin{align*}
        \mu^*(E) + \mu^*(F) \leq \mu^*(E \cup F).
    \end{align*}
    Since the other direction is obvious, then $\mu^*(E) + \mu^*(F) = \mu^*(E \cup F)$.
\end{enumerate}
Thus, $\mu^*$ is a metric outer measure.
\end{proof}

\medskip

\begin{exercise}
Show that
\begin{enumerate}[label=(\alph*)]
    \item If $\H^s(E) < \infty$, then $\H^t(E) = 0$ for all $t > s$;
    \item If $\H^s(E) > 0$, then $\H^t(E) = \infty$ for all $0 < t < s$.
\end{enumerate}
\end{exercise}
\begin{proof}
It has been proved in Exercise \ref{exercise_151}, section \ref{hausdorff_measure}.
\end{proof}

\medskip

\begin{exercise}
Let $f:\mathbb{R}^n\to\mathbb{R}^n$ be a homeomorphism. Prove that $A\subset\mathbb{R}^n$ is Borel if and only if $f(A)$ is Borel.
\end{exercise}
\begin{proof}
It has been proved in Proposition \ref{proposition_110}.
\end{proof}

\medskip

\begin{exercise}
Let $\CC \subset [0,1]$ be the ternary Cantor set. Prove that $\dim_{H}(\CC) \leq \log 2/\log 3$.
\end{exercise}
\begin{proof}
Cantor set $C$ can be covered by $1$ segment of length $1$, or $2$ segments of length $1/3$, or $4$ segments of length $1/3^2$, $\cdots$, or $2^n$ segments of length $3^{-n}$ and so on. Let $s = \log 2/\log 3$, then by the definition of $\H^s_{\varepsilon}$, we have
\begin{align*}
    \H^s_{3^{-n}}(C) \leq \frac{\omega_s}{2^s} 2^n \left(3^{-n}\right)^s.
\end{align*}
Note that $3^s = \left(e^{\log 3}\right)^{\log 2/\log 3} = 2$ and $\left(3^{-n}\right)^3 = 2^{-n}$, then 
\begin{align*}
    \H^s_{3^{-n}}(C) \leq \frac{\omega_s}{2^s} 2^n \left(3^{-n}\right)^s = \frac{\omega_s}{2^s},
\end{align*}
letting $n \to \infty$ implies $\H^s(C) \leq \omega_s / 2^s$. Thus, $\dim_{H}(\CC) \leq \log 2/\log 3$.
\end{proof}

\medskip

\begin{exercise}
Prove that the set of irrational numbers $\mathbb{R} \setminus \mathbb{Q}$ is a Borel set.
\end{exercise}
\begin{proof}
Since $\mathbb{Q}$ is countable, then $\mathbb{Q}$ can be written as $\mathbb{Q} = \bigcup^\infty_{i \in \mathbb{Q}}\{i\}$, where $\{i\}$ is closed. Then,
\begin{align*}
    \mathbb{R} \setminus \mathbb{Q} & = \mathbb{R} \setminus \bigcup^\infty_{i \in \mathbb{Q}}\{i\} = \bigcap_{i \in \mathbb{Q}} \left(\mathbb{R} \setminus \{i\} \right),
\end{align*}
where $\mathbb{R} \setminus \{i\}$ are open. Thus, $\mathbb{R} \setminus \mathbb{Q}$ is $G_\delta$ set and therefore Borel set.
\end{proof}

\medskip

\begin{exercise}
Let $\mathcal{C}$ be the collection of all uncountable sets in $\mathbb{R}$. Prove that the $\sigma$-algebra generated by $\mathcal{C}$ is the power set $\mathcal{P}(\mathbb{R})$.
\end{exercise}
\begin{proof} 
~\begin{enumerate}[label=(\alph*)]
    \item First, $\sigma(\mathcal{C}) \subset \mathcal{P}(\mathbb{R})$.
    
    \item Second, for any set $A \subset \mathbb{R}$. If $A$ is uncountable, then clearly, $A \in \sigma(\mathcal{C})$. If $A$ is countable, then there exists an uncountable set $C \subset \mathbb{R}$ such that $A \cap C = \emptyset$. Then, $A \cup C$ is an uncountable subset, and hence $A = (A \cup C) \setminus C \in \sigma(\mathcal{C})$. Therefore, $\mathcal{P}(\mathbb{R}) \subset \sigma(\mathcal{C})$.
\end{enumerate}
Thus, $\mathcal{P}(\mathbb{R}) = \sigma(\mathcal{C})$.
\end{proof}

\medskip

\begin{exercise}
Prove that the $\sigma$-algebra generated by intervals $(a,\infty)$, $a\in\mathbb{R}$ coincides with the $\sigma$-algebra of all Borel sets $\BB(\mathbb{R})$.
\end{exercise}
\begin{proof}
~\begin{enumerate}
    \item[(a)] Denote $\sigma(G)$ by the $\sigma$-algebra generated by intervals $(a,\infty)$, $a\in\mathbb{R}$. For any interval $(a,\infty)$, we have
    \begin{align*}
        (a,\infty) = \bigcup^\infty_{i=1} \left[a + \frac{1}{i}, a + i\right] \coloneqq \bigcup^\infty_{i=1} F_i, i = 1,2,\cdots,
    \end{align*}
    and since $F_i$ are closed, then $(a,\infty)$ is Borel, and hence $\sigma(G) \subset \BB(\mathbb{R})$.
    
    \item[(b)] For any open set $(a,b)$, $a < b$, we need to show that $(a,b) \in \sigma(G)$. Since $(a,\infty) \in \sigma(G)$, then $(-\infty,a] \in \sigma(G)$, and hence $(a,b] = (a,\infty) \setminus ((a,\infty) \setminus (-\infty,b]) \in \sigma(G)$. Now for any open set $(a,b)$, we have
    \begin{align*}
        (a,b) = \bigcup^\infty_{i=1} \left(a,b-\frac{1}{i}\right], i = 1,2,\cdots,
    \end{align*}
    and hence $(a,b) \in \sigma(G)$. Since $\BB(\mathbb{R})$ is the $\sigma$-algebra generated by all open sets, therefore, $\BB(\mathbb{R}) \subset \sigma(G)$.
\end{enumerate}
Thus, $\BB(\mathbb{R}) = \sigma(G)$.
\end{proof}

\medskip

\begin{exercise}
Prove that the intersection of two $F_\sigma$-sets is $F_\sigma$.
\end{exercise}
\begin{proof}
Suppose $A$ and $B$ are two $F_\sigma$ sets, such that $A = \bigcup^\infty_{i=1} A_i$ and $B = \bigcup^\infty_{i=1} B_i$, where $A_i, B_i$ are all closed sets. Then,
\begin{align*}
    A \cap B = \left(\bigcup^\infty_{i=1} A_i\right) \cap \left(\bigcup^\infty_{i=1} B_i\right) = \bigcup^\infty_{i=1} \bigcup^\infty_{j=1} (A_i \cap B_j),
\end{align*}
where $A_i \cap B_j$ are also closed, hence $A \cap B$ is a countable union of closed sets. Thus, $A \cap B$ is $F_\sigma$-set.
\end{proof}

\medskip

\begin{exercise}
Prove that the set of rational numbers is not $G_\delta$.
\end{exercise}
\begin{proof}
Suppose $\mathbb{Q}$ is $G_\sigma$, then $\mathbb{Q} = \bigcap^\infty_{i=1} G_i$, where $G_i$ is dense in $\mathbb{R}$. Let $\widetilde{G_i} = \mathbb{R} \setminus \{q_i\}, q_i \in \mathbb{Q}$ for $i = 1,2,\cdots$, which are open and dense. Now,
\begin{align*}
    \left(\bigcap^\infty_{i=1}G_i\right) \cap \left(\bigcap^\infty_{i=1}\widetilde{G_i}\right) = \mathbb{Q} \cap (\mathbb{R} \setminus \mathbb{Q}) = \emptyset,
\end{align*}
and this is a contradiction to Baire Category theorem.
\end{proof}

\medskip

\begin{exercise}
Prove that if $0 < s < \infty$ and $E \subset X$ is a subset of a metric space, then $\mathcal{H}^s(E) = 0$ if and only if $\mathcal{H}_\infty^s(E) = 0$. ({\em Remark}: Recall that $\mathcal{H}^s_\infty(E) = \inf \frac{\omega_s}{2^s} \sum_{i=1}^\infty (\diam A_i)^s$, where the infimum is taken over all coverings $E \subset \bigcup_{i=1}^\infty A_i$, $\diam A_i < \infty$.)
\end{exercise}
\begin{proof}
~\begin{enumerate}
    \item[($\Rightarrow$)] If $\H^s(E) = 0$, then  $\H^s_\infty(E) = 0$ follows easily. Indeed, \begin{align*}
        \H^s_\infty(E) \leq \inf \frac{\omega_s}{2^s} \sum_{i=1}^\infty (\diam A_i)^s = \H^s_\varepsilon(E) \xrightarrow[]{\varepsilon \to 0} \H^s(E),
    \end{align*}  
    since the infimum is taken over all coverings $E \subset \bigcup^\infty_{i=1} A_i$ with $\diam A_i < \varepsilon$, which is a smaller collection of coverings.
    
    \item[($\Leftarrow$)] If $\H^s_\infty(E) = 0$, then for any $\varepsilon > 0$, there exists a covering $\{G^\varepsilon_i\}$ of $E$ with $\diam G^\varepsilon_i < \varepsilon^{1/s}, i = 1,2,\cdots$. Otherwise, there exists $\varepsilon > 0$ such that for all coverings $\{G_i\}$ of $E$ with $\diam G_i \geq \varepsilon^{1/s}$, $\H^s_\infty(E) = 0$, which is a contradiction. Letting $\varepsilon \to 0$ implies
    \begin{align*}
        \H^s(E) = \lim_{\varepsilon^{1/s} \to 0}\H^s_{\varepsilon^{1/s}}(E) \leq \lim_{\varepsilon^{1/s} \to 0} \frac{\omega_s}{2^s} \sum^\infty_{i=1} (\diam G^\varepsilon_i)^s = 0, \quad \diam G^\varepsilon_i < \varepsilon^{1/s}.
    \end{align*}
\end{enumerate}
\end{proof}

%https://www.math.stonybrook.edu/~rdhough/mat639-spring17/lectures/lecture20.pdf
\medskip

\begin{exercise}\label{exeercise_112}
Show an example of a set such that $\mathcal{H}^s(E) > \mathcal{H}^s_\varepsilon(E)$ for some $s > 0$ and all $\varepsilon > 0$.
\end{exercise}
\begin{proof}
For the interval $(0,1)$, we know that $\mathcal{H}^1((0,1)) = 1 - 0 = 1$. Then, for any $0 < s < 1$, $\mathcal{H}^s((0,1)) = \infty$. For a fixed $s$ and any $\varepsilon > 0$, there always exists a finite covering $(0,1) \subset \bigcup^k_{i=1} A_i$, such that $\diam A_i < \varepsilon$. Thus, for any $\varepsilon> 0$,
\begin{align*}
    \mathcal{H}^s_{\varepsilon}((0,1)) = \frac{\omega_s}{2^s} \sum^k_{i=1} (\diam A_i)^s < \infty = \mathcal{H}^1((0,1)).
\end{align*}
\end{proof}

\begin{proof}[Second Possible Proof of Exercise \ref{exeercise_112}]\footnote{I try to use the definition of Hausdorff measure $\mathcal{H}^s$, but currently I do not know if this proof is valid.} Let $E = [0,1] \times \{1,2,\cdots, n\} \in \mathbb{R}^2$. Then,
\begin{align*}
    \mathcal{H}^1_{1/n} = \frac{\omega_1}{2} n^2 \frac{1}{n} = \frac{\omega_1}{2} n,
\end{align*}
and $\mathcal{H}^1(E) = \lim_{n\to\infty} \mathcal{H}^1_{1/n} = \infty$.
\end{proof}

\medskip

\begin{exercise}
Prove that if $X$ is a locally compact separable metric space, then there are open sets $\{U_i\}_{i=1}^\infty$ such that
$$
X=\bigcup_{i=1}^\infty U_i
\quad
\text{and}
\quad
\text{$\overline{U}_i$ is compact.}
$$
\end{exercise}
\begin{proof}
Since $X$ is locally compact, then for every point $x \in X$, there is a compact set $K_x$ that contains an open neighborhood $U_x$ of $x$, that is $x \in U_x \subset K_x$. Also, since $X$ is separable, then it has a countable base $\{V_n \}^\infty_{n=1}$. Then for any $x \in U_x \subset X$, there is a $n_x$ such that $x \in V_{n_x} \subset U_x \subset K_x$. Taking closure yields $x \in \overline{V}_{n_x}$, where $\overline{V}_{n_x}$ is compact since $\overline{V}_{n_x} \subset \overline{K}_x = K_x$. Thus, $$X = \bigcup_{n_x} V_{n_x},$$ which is a countable union since $\{n_x\}$ is a subsequence of $\{n\}$ and $\overline{V}_{n_x}$ is compact.
\end{proof}

\medskip

\begin{exercise}\label{exercise_115}\footnote{The problem comes from Chapter 1, {\em Real Analysis: Measure Theory, Integration, and Hilbert Spaces} \cite{8}.} Does there exist an enumeration $\{ r_n\}^\infty_{n=1}$ of the rationals, such that the complement of
$$
\bigcup^\infty_{n=1}\biggr(r_n-\frac{1}{n},r_n+\frac{1}{n}\biggr)
$$
in $\mathbb{R}$ is non-empty? ({\bf Hint:} Find an enumeration where the only rationals outside of a fixed bounded interval take the form $r_n$, with $n = m^2$ for some integer $m$.)
\end{exercise}
\begin{proof}\cite{19}
Fixed an irrational number $\alpha$, we want to build a new enumeration $\{q_n\}^\infty_{n=1}$ of $\mathbb{Q}$ such that for any $n \in \mathbb{Z}^+$, $\alpha \notin \left(q_n - 1/n, q_n + 1/n\right)$.

Let $n_1 = \min\{n \in \mathbb{Z}^+ \,:\, \left|r_n - \alpha\right| \geq 1 \}$ and let $q_1 = r_{n_1}$, then clearly, $\alpha \notin (q_1 - 1, q_1 + 1)$. Let $Z_1 = \mathbb{Z}^+ \setminus \{n_1\}$. Continue this process and now suppose we found $q_1,\cdot,q_m$ and defined $Z_m$. Let 
\begin{align*}
    n_{m+1} = \min \left\{n \in Z_m \,:\, \left|r_n - \alpha\right| \geq \frac{1}{m+1}\right\},
\end{align*}
and set $q_{m+1} = r_{n_{m+1}}$, also, $Z_{m+1} = Z_m\setminus \{n_{m+1}\}$. Clearly, $q_{m+1}$ is different from $q_1,q_2,\cdot,q_m$, and 
\begin{align*}
    \alpha \notin \left(q_n - \frac{1}{n}, q_n + \frac{1}{n}\right).
\end{align*}
At each stage we took the first available rational in the original enumeration, then every rational number is eventually enumerated as $q_n$ for some $n \in \mathbb{Z}^+$. Thus, 
\begin{align*}
    \alpha \notin \bigcup^\infty_{n=1}\left(q_n - \frac{1}{n}, q_n + \frac{1}{n}\right).
\end{align*}
\end{proof}

\medskip

\begin{proof}[Second Proof of Exercise \ref{exercise_115}]\cite{20} 
Denote all rational numbers in $[0,1]$ by $\{p_n\}$ and those in $\mathbb{R} \setminus [0,1]$ by $\{q_n\}$. Now let's construct $\{r_n\}$ as follows: if $n$ is a square of some integer, we take the next one in $\{q_n\}$, otherwise we take the next one in $\{q_n\}$. Clearly, 
\begin{align*}
    \bigcup^\infty_{n=1}\left(r_n - \frac{1}{n}, r_n + \frac{1}{n}\right) \subset [-1,2] \cup \bigcup^\infty_{n=1} \left(r_{n^2} - \frac{1}{n^2}, r_{n^2} + \frac{1}{n^2}\right),
\end{align*}
and thus 
\begin{align*}
    \L_n\left(\bigcup^\infty_{n=1}\left(r_n - \frac{1}{n}, r_n + \frac{1}{n}\right)\right) \leq 3 + \sum^\infty_{n=1} \frac{2}{n^2} < \infty,
\end{align*}
which yields the statement.
\end{proof}

\medskip

\begin{exercise}\label{exe_116}
Let $A$ be the subset of $[0,1]$ which consists of all numbers which do not have the digit 4 appearing in their decimal expansion. Find $\mathcal{L}_1(A)$.
\end{exercise}
\begin{proof}\cite{22}
Let $x \in A$. For any $\varepsilon > 0$, there exists $N > 0$ such that $k/10^N < \varepsilon$ for $k = 1,2,\cdots,9$. Then, $x + k/10^N \in B(x,\varepsilon)$ for all $k = 1,2,\cdots,9$. Now, we can find a $k$ such that the $N$-th digit in the decimal expansion of $x + k/10^N$ is $4$. Therefore, $x$ is a singleton set and $A$ is an uncountable union of singleton sets. Thus, $\mathcal{L}_1(A) = 0$.
\end{proof}

\medskip

\begin{proof}[Second Proof of Exercise \ref{exe_116}]
Let $A_n = \{x \in [0,1] \,:\, \text{first $n$ digits of } x \neq 4 \}$. Clearly, $A_1 \supset A_2 \supset A_3 \supset \cdots$ is a decreasing sequence of sets. Also, since $\L_1(A_1) < \L_1([0,1]) = 1$, then by Theorem \ref{theorem_16}(g), we have
\begin{align*}
    L_1(A) = \L_1\left(\bigcap^\infty_{i=1} A_i\right) = \lim_{i\to\infty} \L_1(A_i).
\end{align*}
Observe that $A_1 = [0,0.4) \cup [0.5,1]$ and then $\L_1(A_1) = 9/10$. Consider $A_2$, then
\begin{align*}
    A_2 = A_1 \setminus \bigcup^8_{k=1,k\neq 4}[0.k4,0.k5),
\end{align*}
and note that $\L_1(A_2) = 9/10 \L_1(A_1)$. By induction, we have
\begin{align*}
    L_1(A) = \L_1\left(\bigcap^\infty_{i=1} A_i\right) = \lim_{i\to\infty} \L_1(A_i) = \lim_{i\to\infty} \frac{9^i}{10^i} = 0.
\end{align*}
\end{proof}

\medskip

\begin{exercise}\footnote{This is called Borel-Cantelli Lemma, and a simple proof is provided in\cite{22}.}
Suppose $\{E_k\}^\infty_{k=1}$ is a countable family of measurable subsets of $\mathbb{R}^n$ such that
$\sum^\infty_{k=1} \mathcal{L}_n(E_k)<\infty$ and let
$$
E=\{ x\in\mathbb{R}^n:x\in E_k\text{ for infinitely many }k\}.
$$
Show that $E$ is measurable and that $\mathcal{L}_n(E)=0$.
\end{exercise}
\begin{proof}
Note that 
\begin{align*}
    E = \bigcap_{k\in \mathbb{N}} \bigcup_{j\geq k} E_j.
\end{align*}
Let $A_k \coloneqq \bigcup_{j\geq k} E_j$ and $A_k$ is measurable since it is union of measurable sets, and hence $E$ is also measurable.

Since $\sum^\infty_{k=1} \mathcal{L}_n(E_k) < \infty$, then for any $\varepsilon > 0$, there exists $N > 0$ such that for all $K > N$,
\begin{align*}
    \sum^\infty_{k=K} \mathcal{L}_n(E_k) < \infty.
\end{align*}
Note that $E \subset \bigcup^\infty_{k=K} E_k$, then
\begin{align*}
    \mathcal{L}_n(E) \leq \mathcal{L}_n\left(\bigcup^\infty_{k=K} E_k\right) \sum^\infty_{k=K} \mathcal{L}_n(E_k) < \infty.
\end{align*}
Thus, $\mathcal{L}_n(E)=0$.
\end{proof}

\medskip

\begin{exercise}
Let $K\subset\mathbb{R}^2$ be a compact set and let $\pi:\mathbb{R}^2\to\mathbb{R}$ be the orthogonal projection onto the $x$-axis. Prove that if $\pi(K)=[0,1]$, then
$\H^1(K)\geq 1$.
\end{exercise}
\begin{proof}
We have $\H^1([0,1]) = 1$ and it suffices to prove that $\H^1(K) \geq \H^1([0,1])$. For any $x_1, x_2 \in K \subset \mathbb{R}^2$, since $\pi$ is an orthogonal projection, then $\left|x_1,x_2\right| \geq \left|\pi(x_1),\pi(x_2)\right|$, which implies for any open set $A \subset \mathbb{R}^2$, $\diam A \geq \diam \pi(A)$. 

For any $\varepsilon > 0$, suppose that the covering $K \subset \bigcup^\infty_{i=1} A_i$ satisfies \begin{align*}
    \H^1_{\varepsilon} = \frac{\omega_1}{2} \sum^\infty_{i=1}\diam A_i, \quad \diam A_i < \varepsilon.
\end{align*}
Since $K$ is compact, then $\{A_i\}$ has a finite subcovering $\bigcup^K_{j=1}A_{i_j}$. Then for any $\varepsilon > 0$, $\left\{\pi(A_{i_j})\right\}$ is a covering of $[0,1]$, and then we have
\begin{align*}
    H^1_\varepsilon(K) \geq \frac{\omega_1}{2} \sum^K_{j=1} \diam A_{i_j} \geq \frac{\omega_1}{2} \sum^K_{j=1} \diam \pi(A_{i_j}) \geq \inf \frac{\omega_1}{2} \sum^\infty_{i=1} \diam B_i = \H^1_{\varepsilon}([0,1]),
\end{align*}
where $\{B_i\}$ is any covering of $[0,1]$ with $\diam B_i < \varepsilon$. Letting $\varepsilon \to 0$ yields the result.
\end{proof}

\medskip

\begin{exercise}
You can use the fact that if $S^1\subset\mathbb{R}^2$ is the unit circle, then $\H^1(S^1)=2\pi$.
Assume that $\phi:S^1\to\mathbb{R}^2$ is a map such that for some $L\geq 1$ and $0<s<1$ we have
$$
\frac{1}{L}\left|x-y\right|^s\leq \left|\phi(x)-\phi(y)\right|\leq L\left|x-y\right|^s,
\quad
\text{for all $x,y\in S^1$.}
$$
Find $\dim_{H}(\phi(S^1))$.
\end{exercise}
\begin{proof}
We claim that $\dim_{H}(\phi(S^1)) = 1/s$. It suffices to prove that $\H^1(\phi(S^1)) = 0$.

For any $\varepsilon > 0$, there is a covering $S^1 \subset \bigcup^\infty_{i=1} A_i$, with $\diam A_i < \varepsilon$. And $\{\phi(A_i)\}$ is a covering of $\{\phi(S^1)\}$ with $\varepsilon^s/L \leq \diam \phi(A_i) \leq L \varepsilon^s$. Then we have
\begin{align*}
    \frac{\varepsilon^s}{L}\frac{\omega_1}{2} \sum^\infty_{i=1} \diam A_i \leq\frac{\omega_1}{2} \sum^\infty_{i=1} \diam \phi(A_i) \leq L \varepsilon^s \frac{\omega_1}{2} \sum^\infty_{i=1} \diam A_i,
\end{align*}
taking infimum of all coverings $\{a_i\}$ with $\diam A_i < \varepsilon$ yields
\begin{align*}
    \frac{\varepsilon^s}{L} \H^1_{\varepsilon}(S^1) \leq \H^1_{L\varepsilon^s}(\phi(S^1)) \leq L \varepsilon^s \H^1_{\varepsilon}(S^1).
\end{align*}
Now letting $\varepsilon \to 0$ gives $\H^1(\phi(S^1)) = 0$, and thus $\dim_{H}(\phi(S^1)) = 1/s$.
\end{proof}

\medskip

\begin{exercise}\label{exe_120}
Regard $\mathbb{R}$ as a subset of $\mathbb{R}^2$ (the $x$-axis). Prove that $A\subset\mathbb{R}$ is Borel as a subset of $\mathbb{R}$ if and only if it is Borel as a subset of $\mathbb{R}^2$.
\end{exercise}
\begin{proof}
~\begin{enumerate}
    \item[($\Rightarrow$)] Let $\pi:\mathbb{R}^2 \to \mathbb{R}$ be an orthogonal projection, defined as $\pi(x,y) = x$. Clearly, $\pi$ is continuous, then $\pi$ is a Borel mapping. For any Borel set $E \subset \mathbb{R}$, we have $\pi^{-1}(E) = E \times \mathbb{R}$, which is Borel. Since $E = (E \times \mathbb{R}) \cap (\mathbb{R} \times \{0\})$ and both sets are Borel, then $E$ is Borel.
    
    \item[($\Leftarrow$)] Let $i:\mathbb{R} \to \mathbb{R}^2$ be an embedding, defined as $i(x) = x$, the identify. Clearly, $i$ is continuous, then $i$ is a Borel mapping. For any set $E \subset \mathbb{R}$ and $E$ is Borel in $\mathbb{R}^2$, then $i^{-1}(E) = E \cap (\mathbb{R} \times \{0\}) = E$ is Borel in $\mathbb{R}$.\footnote{My first thought is that let $f(x): \mathbb{R} \to \mathbb{R}^2$ be defined as $f(x) = (x,0) \in \mathbb{R}^2$. Clearly, $f$ is a Borel mapping. By Definition \ref{def_borel_map}, $f^{-1}(E)$ is Borel if $E$ is Borel. This is similar to the proof in\cite{23}.}
\end{enumerate}
\end{proof}


\begin{proof}[Second Proof of Exercise \ref{exe_120}]
~\begin{enumerate}
    \item[($\Rightarrow$)] Let $\mathcal{C}$ be the $\sigma$-algebra generated by sets of form $F \times \{0\}$, where $F \subset \mathbb{R}$ is closed. Clearly, $\mathcal{C} \subset \mathfrak{B}(\mathbb{R}^2)$ since the sets that generating $\mathcal{C}$ are all closed. Suppose that $A \subset \mathbb{R}$ is Borel, since $\mathfrak{B}(\mathbb{R})$ is generated by closed subsets of $\mathbb{R}$, then $A \times \{0\} \in \mathcal{C} \subset \mathfrak{B}(\mathbb{R}^2)$. Hence, $A$ is Borel in $\mathbb{R}^2$.
    
    \item[($\Leftarrow$)] If $A$ is Borel in $\mathbb{R}^2$, then $A$ can be a slice of Borel set $E$ in $\mathbb{R}^2$. Assume $E$ is a Borel subset of $\mathbb{R}^2$, let $\widetilde{E} = \{x \in \mathbb{R}\,:\, (x,y) \in E\}$ be a slice of $E$ in $\mathbb{R}$. Let $\mathcal{F} = \{E \subset \mathbb{R}^2\,:\, \widetilde{E}\,\,\text{is Borel in $\mathbb{R}$}\}$. Clearly, $\mathcal{F}$ is a $\sigma$-algebra.
    
    It remains to prove that $\mathcal{F}$ contains all open sets. If $U \subset \mathbb{R}^2$ is open, then it is easy to see that $\widetilde{U}$ is also open. Indeed, if $x \in \widetilde{U}$, then there is an open ball containing $(x,y) \in U$, and the slice of this open ball is an open interval containing $x$.\footnote{This part is based on Milo Brandt's proof \cite{24}.} Therefore, $\widetilde{U}$ is Borel and hence $\mathcal{F}$ contains open sets. Thus, $\mathfrak{B}(\mathbb{R}^2) \subset \mathcal{F}$, and hence $E \subset \mathcal{F}$, which implies its slice $A$ is Borel.
\end{enumerate}
\end{proof}

\medskip

\begin{exercise}
Prove that there is an $\mathcal{L}_2$ measurable set $A\subset\mathbb{R}^2$ that is not Borel.
\end{exercise}
\begin{proof}
Let $E \subset [0,1]$ be nonmeasuarable, and since $[0,1] \subset \mathbb{R}$, then $\mathcal{L}_2([0,1]) \leq \mathcal{L}_2(\mathbb{R}) = 0$, and hence $\mathcal{L}_2(E) = 0$. Therefore, $E$ is $\mathcal{L}^*_2$-measurbale and hence Lebesgue measurable. However, by the previous problem, $E$ is not a Borel set.
\end{proof}

\medskip

\begin{exercise}
Prove that if $A\subset\mathbb{R}^n$ and $B\subset\mathbb{R}^m$ are $\mathcal{L}_n$ and $\mathcal{L}_m$ measurable respectively, then $A\times B$ is $\mathcal{L}_{n+m}$ measurable.
\end{exercise}
\begin{proof}\cite{25}
Since $A\subset\mathbb{R}^n$ and $B\subset\mathbb{R}^m$ are measurable, then for any $\varepsilon > 0$, there exist open sets $G_1 \subset \mathbb{R}^n,G_2 \subset \mathbb{R}^m$ and closed sets $F_1 \subset \mathbb{R}^n,F_2 \subset \mathbb{R}^m$ such that $F_1 \subset A \subset G_1$, $F_2 \subset B \subset G_2$ and
\begin{align*}
    \mathcal{L}_n(G_1\setminus F_1) < \frac{\varepsilon}{2M},\quad \mathcal{L}_m(G_2\setminus F_2) < \frac{\varepsilon}{2M},
\end{align*}
where $M = \max\{\mathcal{L}_n(F_1), \mathcal{L}_m(F_2)\}$. 

Now, $F_1 \times F_2 \subset A \times B \subset G_1 \times G_2$ and $F_1 \times F_2$ is closed, $G_1 \times G_2$ is open. Since
\begin{align*}
    G_1 \times G_2 & = \left(G_1 \times F_2\right) \cup (G_1 \times (G_1 \setminus F_2)) \\
    & = \left(F_1 \times F_2\right) \cup \left((G_1 \setminus F_1) \times F_2\right) \cup (G_1 \times (G_1 \setminus F_2)),
\end{align*}
then 
\begin{align*}
    \mathcal{L}_{n+m}\left((G_1 \times G_2) \setminus (F_1 \times F_2)\right) = \mathcal{L}_{n+m}\left((G_1 \setminus F_1) \times F_2\right) + \mathcal{L}_{n+m} (G_1 \times (G_1 \setminus F_2)) < \varepsilon.
\end{align*}
Thus, $A\times B$ is $\mathcal{L}_{n+m}$ measurable.
\end{proof}

\medskip

\begin{exercise}
Prove that if $A\subset\mathbb{R}^n$ satisfies $\mathcal{L}_n(A)=0$ and $B\subset\mathbb{R}^m$ is any set, then
$\mathcal{L}_{n+m}(A\times B)=0$.
\end{exercise}
\begin{proof}
Since $\mathcal{L}_n(A)=0$, then for any $\varepsilon > 0$, there exists a family of closed intervals $\{P_i\}$ in $\mathbb{R}^n$ such that 
\begin{align*}
    A \subset \bigcup^\infty_{i=1} P_i, \quad \sum^\infty \mathcal{L}_n(P_i) < \varepsilon.
\end{align*}
Also, for any $B \subset \mathbb{R}^m$, $B$ is contained in a closed interval $P \subset \mathbb{R}^m$. Then, $A \times B \subset \bigcup^\infty_{i=1}(P_i \times P)$, and hence
\begin{align*}
    \mathcal{L}^*_{m+n}(A \times B) \leq \sum^\infty_{i=1} \mathcal{L}^*_{m+n}(A_i \times P) = \sum^\infty_{i=1} \mathcal{L}_{m}(A_i) \mathcal{L}_{n}(P) = \mathcal{L}_{n}(P) \varepsilon.
\end{align*}
Thus, $\mathcal{L}_{n+m}(A\times B)=0$.
\end{proof}

\medskip

\begin{exercise}
Prove that the graph
$$
G_f=\{(x,f(x)):\, y=f(x)\}\subset\mathbb{R}^2
$$
of a continuous function $f:\mathbb{R}\to\mathbb{R}$ satisfies $\mathcal{L}_2(G_f)=0$.
\end{exercise}
\begin{proof}\cite{26}
It suffices to show that $\mathcal{L}_2(G_f)=0$ on the interval $[0,1]$. Since $f$ is continuous, then $f$ is uniformly continuous on compact set $[0,1]$, that is, for any $\varepsilon > 0$, there is a $\delta > 0$ such that for any $x, y \in [0,1]$, if $\left|x - y\right| < \delta$, $\left|f(x) - f(y)\right| < \varepsilon/2$. Now, let $n > 1/\delta$ and divide $[0,1]$ into $n$ intervals with the same length $1/n < \delta$. Then, for the graph of $f$ over $[0,1]$, we have
\begin{align*}
    G_f|_{[0,1]} \subset \bigcup^{n-1}_{k=0} \left[\frac{k}{n},\frac{k+1}{n}\right] \times \left[f\left(\frac{k}{n}\right) - \frac{\varepsilon}{2},f\left(\frac{k}{n}\right) + \frac{\varepsilon}{2}\right] \coloneqq \bigcup^{n-1}_{k=0} P_i,
\end{align*}
and hence
\begin{align*}
    \mathcal{L}_2\left(G_f|_{[0,1]}\right) \leq \sum^{n-1}_{k=0} \mathcal{L}_2\left(P_i\right) = n \cdot \frac{\varepsilon}{n} = \varepsilon.
\end{align*}
Thus, $\mathcal{L}_2(G_f)=0$ on $[0,1]$ and hence $\mathcal{L}_2(G_f)=0$ on $\mathbb{R}$.
\end{proof}

\medskip

\begin{exercise}
Prove that if $X$ is a compact metric space, then for any $\varepsilon>0$ and any $s>0$, $\H^s_\varepsilon(X)<\infty$.
\end{exercise}
\begin{proof}
Let $X = \bigcup^\infty_{i=1} A_i$ be any covering of $X$ with $\diam A_i < \varepsilon$. Since $X$ is compact, then the covering $\{A_i\}^\infty_{i=1}$ has finite subcovering $\{A_{i_j}\}^K_{j=1}$, and hence
\begin{align*}
    \mathcal{H}^s_{\varepsilon}(X) \leq \frac{\omega_s}{2^s} \sum^K_{j=1} \diam A_{i_j} < \frac{\omega_s}{2^s} K \varepsilon < \infty.
\end{align*}
\end{proof}

\medskip

\begin{exercise}
Prove that if $X$ is a bounded metric space and $\varepsilon>\diam X$, then for any $s>0$, $\H^s_\varepsilon(X)<\infty$.
\end{exercise}
\begin{proof}
Since $X$ is bounded and $\diam X < \varepsilon$, $X$ is contained in an open ball centered as any point $x \in X$ with radius $\varepsilon$, denoted by $B(x,\varepsilon)$. Thus,
\begin{align*}
    \mathcal{H}^s_{\varepsilon}(X) \leq \frac{\omega_s}{2^s} \diam B(x,\varepsilon) = \frac{\omega_s}{2^{s-1}} \varepsilon < \infty.
\end{align*}
\end{proof}

\medskip

\begin{exercise}
Prove that there is a bounded separable metric space such that for any $0<\varepsilon<\diam X$ and any $s>0$, $\H^s_\varepsilon(X)=\infty$. ({\bf Hint:} Subset of $\ell^\infty$.)
\end{exercise}
\begin{proof}
Note that $\ell^\infty = \left\{x = (x_1,x_2,\cdots) \,:\, \sup_i \left|x_i\right| < \infty \right\}$. Let
\begin{align*}
    S = \left\{(x_1, x_2, \cdots) \,:\, x_1 \in [0,1],\, x_j = 1\,\, \text{for some $j$},\, x_i = 0, i \neq 1,j \right\},
\end{align*}
and the distance between any two elements in $S$ is $1$. Consider the set 
\begin{align*}
    T = S \cap \{(x_1, x_2, \cdots) \,:\, x_1 \in \mathbb{Q} \},
\end{align*}
then the cardinality of $T$ is $|\mathbb{Q}| |\mathbb{N}| = \aleph_0^2$, and hence $T$ is countable. Observe that $T$ is dense in $S$, then $S$ is separable. 

Let $e_i = (0,\cdots,0,1,0,\cdots)$, where $1$ only appears at $i$th entry. Define $P_i = [0,1] \times e_i$, clearly, $S = \bigcup^\infty_{i=1} P_i$. Now for any $0 < \varepsilon < \diam X$ and $i \in \mathbb{N}$, let $P_i \subset \bigcup^\infty_{j=1} A_{i_j}$ be an open covering of $P_i$ with $\diam A_{i_j} < \varepsilon$. Let $\pi$ be the orthogonal projection defined by $\pi(x) = x_1, x \in \ell^\infty$. Observe that for any $x, y \in S$,
\begin{align}\label{prob_27_equ1}
    d_\infty(x,y) \geq \left|r - s\right| = d(\pi(x),\pi(y)).
\end{align}
Clearly, $\bigcup^\infty_{j=1} \pi(A_{i_j})$ is an open covering of $[0,1]$ since $\pi(S) = [0,1]$ and by \eqref{prob_27_equ1}, $\diam \pi(A_{i_j}) < \varepsilon$ for all $i$. Choose any $s > 0$, we have
\begin{align*}
    \frac{\omega_s}{2^s} \sum^\infty_{j=1} (\diam A_{i_j})^s \geq \frac{\omega_s}{2^s} \sum^\infty_{j=1} (\diam \pi(A_{i_j}))^s \geq \mathcal{H}^s_{\varepsilon}(\pi(S)) = \mathcal{H}^s_{\varepsilon}([0,1]) > 0.
\end{align*}
Thus, for $i \neq j$, $\operatorname{dist}(P_i,P_j) = 1$, and since $\mathcal{H}^s$ is a measure on Borel sets, we have
\begin{align*}
    \mathcal{H}^s_\varepsilon(S) = \sum^\infty_{i=1} \mathcal{H}^s_\varepsilon(P_i) \geq \sum^\infty_{i=1} \mathcal{H}^s_{\varepsilon}([0,1]) = \infty.
\end{align*}
\end{proof}

\medskip

\begin{proof}[Second Thought]
If we take
\begin{align*}
    S = \left\{(x_1, x_2, \cdots) \,:\, 0 \leq x_i \leq \frac{1}{i} \right\},
\end{align*}
and let
\begin{align*}
    T = S \cap \mathbb{Q}^n,
\end{align*}
clearly, $T$ is dense in $S$. $T$ is countable, since $T$ is a countable union of countable sets, that is $T = \bigcup^\infty_{i=1} P_i$, where
\begin{align*}
    P_i = \left\{(x_1, x_2, \cdots) \,:\, 0 \leq x_i \leq \frac{1}{i}; x_j = 0, j \neq i \right\}.
\end{align*}
\end{proof}

\medskip

\begin{exercise}
Let $A\subset [0,1]$ be a measurable set of positive measure. Show that there exist two points $x,y\in A$, $x\neq y$ such that $x-y$ is a rational number. (Prove it directly without using Steinhaus theorem).
\end{exercise}
\begin{proof}
Let $Q = \mathbb{Q} \cap [0,1]$. Clearly $Q$ is countable and can be represented by $Q = \{q_1, q_2, \cdots\}$. Let $A_n = A + \{q_n\} = \{x + q_n \,:\, x \in A\}$. Clearly, $\mu(A) = \mu(A_n)$, since $A_n$ is just a translation of $A$, where $\mu$ is Lebesgue measure $\mathcal{L}_1$.

If $A_i \cap A_j \neq \emptyset$ for some $i,j \in \mathbb{N}$, then there exists $x, y \in A$, $x \neq y$ such that
\begin{align*}
    x + q_i = y + q_j \in A_i \cap A_j,
\end{align*}
which implies $x - y = q_j - q_i \in \mathbb{Q}$. 

Hence it suffices to show that $A_i \cap A_j \neq \emptyset$ for some $i,j \in \mathbb{N}$. Suppose to the contrary that $A_i \cap A_j = \emptyset$ for all $i,j \in \mathbb{N}$. Clearly, $A_n \subset [0,2]$ for all $n \in \mathbb{N}$. Since $A_i$ are pairwise disjoint and $\mu(A) > 0$, 
\begin{align*}
    2 = \mu([0,2]) \geq \mu\left(\bigcup^\infty_{i=1} A_i \right) = \sum^\infty_{i=1} \mu(A_i) = \sum^\infty_{i=1} \mu(A) = \mu(A) \sum^\infty_{i=1} 1 = \infty,
\end{align*}
which is a contradiction.
\end{proof}

\medskip

\begin{exercise}\label{exe_129}
Let $\MM$ be a $\sigma$-algebra of subsets of $X$ that contains countably many nonempty pairwise disjoint sets. Prove that $\MM$ is uncountable.
\end{exercise}
\begin{proof}
Suppose $\{A_1, A_2, \cdots\}$ is a sequence of countable, pairwise disjoint sets, recall that the power set $\mathcal{P}(\mathbb{N})$ of $\mathbb{N}$ is uncountable. We construct an uncountable set of elements of $\mathfrak{M}$ as follows. For each $Y \in \mathcal{P}(\mathbb{N})$, let $B_Y = \bigcup_{y\in Y} A_y$. We claim that $B_Y \neq B_Z$ if $Y \neq Z$. Indeed, if $B_Y = B_Z$, then $\bigcup_{y\in Y} A_y = \bigcup_{z\in Z} A_z$, and for $x \in B_Y, B_Z$, there are $\widetilde{y} \in Y$ and $\widetilde{z} \in Z$ such that $x \in A_{\widetilde{y}}$ and $x \in A_{\widetilde{z}}$. Hence, $A_{\widetilde{y}} = A_{\widetilde{z}}$, since $A_i$ are disjoint. Therefore,e the collection $\{B_Y \,:\, Y \in \mathcal{P}(\mathbb{N})\} \subset \mathfrak{M}$ has the same cardinality as the power set of $\mathbb{N}$. Thus, $\mathfrak{M}$ contains an uncountable subcollection.
\end{proof}

\medskip

\begin{exercise}\label{exe_130}
Prove that if $\MM$ is an infinite $\sigma$-algebra, then it is uncountable.
\end{exercise}
\begin{proof}
Let $S$ be the underlying set on which $\mathfrak{M}$ is a $\sigma$-algebra. The idea is to divide $S$ into two pieces and prove that $\mathfrak{M}$ restricts to an infinite $\sigma$-algebra on one of the two pieces. 

First, if $A \in \mathfrak{M}$ is any subset, then the collection $\mathfrak{M}_A = \{B \cap A \,:\, B \in \mathfrak{M}\}$ is also a $\sigma$-algebra on $A$. Since $\mathfrak{M}$ is infinite, there exists an element $A_1 \in \mathfrak{M}$ with $\emptyset \subset A_1 \subset S$. Clearly, $A_1$ and $A_1^c$ are in $\mathfrak{M}$, and they cover $S$. Moreover,
\begin{align*}
    \mathfrak{M} = \{B_1 \cup B_2 \,:\, B_1 \in \mathfrak{M}_{A_1}, B_2 \in \mathfrak{M}_{A_1^c} \}.
\end{align*}
So, at least one of $\mathfrak{M}_{A_1}$ and $\mathfrak{M}_{A_1^c}$ must be infinite. Without loss of generality, we assume the latter one is infinite. Since any element of $\mathfrak{M}_{A_1^c}$ is an element of $\mathfrak{M}$ that is disjoint from $A_1$, we may repeat this argument with $S$ replaced by $A_1^c$ and with $\mathfrak{M}$ replaced by $\mathfrak{M}_{A_1^c}$. By induction, we construct an infinite sequence $\{A_1, A_2, \cdots\}$ of pairwise disjoint nonempty elements of $\mathfrak{M}$.

Thus, by the previous problem, $\mathfrak{M}$ is uncountable.
\end{proof}

\begin{remark}
Note that the assumptions are weaker than in Problem \ref{exe_129} and clearly Problem \ref{exe_129} follows from Problem \ref{exe_130}.%http://www.math.hawaii.edu/~xander/fa12_471/Prob_HW1.pdf
\end{remark}

\medskip

\section{Integration}

\begin{exercise}
Prove that if $K\subset\mathbb{R}^n$ is compact and $K_\varepsilon=\{x \,:\, \operatorname{dist}(x,K)<\varepsilon\}$, then $\left|K_\varepsilon\right| \to \left|K\right|$ as $\varepsilon\to 0$.
Show an example of a non-compact measurable set such that $\left|K_\varepsilon\right| \not\to \left|K\right|$.
\end{exercise}
\begin{proof}
Let $K_n = \{x \in \mathbb{R}^n \,:\, \operatorname{dist}(x,K) < 1/n\}$, since $K$ is compact, $\bigcap^\infty_{n=1} K_n = K$. And it follows that $\lim_{n\to\infty} \left|K_n\right| = \left|K\right|$.

For non-compact set $K = \mathbb{Q} \subset \mathbb{R}$, $K_\varepsilon = \mathbb{R}$, and clearly $\left|K_\varepsilon\right| = \infty \neq \left|\mathbb{Q}\right| = 0$.
\end{proof}

\medskip

\begin{exercise}
Give an example of a measurable function $f:[0,1]\to [0,\infty)$ (i.e., with finite values) such that for all $0<a<b<1$ we have
$$
\int_a^b f(x)\, dx = \infty.
$$
\end{exercise}
\begin{proof} 
Let $\{q_n\} = \mathbb{Q} \cap [0,1]$ and define
\begin{align*}
    g(x) = \begin{cases}
        \sum^\infty_{n=1} \frac{2^{-n}}{|x - q_n|^{1/2}}, & x \notin \{q_n\}, \\
        0, & x \in \{q_n\}.
    \end{cases}
\end{align*}
Observe that
\begin{align*}
    \int^1_0 \frac{1}{|x - q_n|^{1/2}}\, dx = \int^{q_n}_0 \frac{1}{|x - q_n|^{1/2}}\, dx + \int^1_{1_n} \frac{1}{|x - q_n|^{1/2}}\, dx \leq 4,
\end{align*}
then
\begin{align*}
    \int^1_0 g(x)\,dx \leq 4 \sum^\infty_{n=1} 2^{-n} < \infty,
\end{align*}
which means $g$ is well defined. Clearly $g \in L^1([0,1])$. Now define
\begin{align*}
    f(x) = \begin{cases}
        g^2(x), & g(x) < \infty, \\
        0, & g(x) = \infty.
    \end{cases}
\end{align*}
For any $(a,b) \subset [0,1]$, there is a $q_i \in (a,b)$, then
\begin{align*}
    \int^b_a f(x)\,dx \geq \int^b_a \frac{2^{n-1}}{|x - q_i|} =  \infty.
\end{align*}
\end{proof}

\medskip

\begin{exercise}
Prove that if $f:X\to [0,\infty]$ is measurable and $\int_Xf\,d\mu<\infty$, then
$\mu(\{x \in X \,:\, f(x) = +\infty\})=0$.
\end{exercise}
\begin{proof} 
Let $E = \{x \in X \,:\, f(x)=+\infty\}$ and suppose to the contrary $\mu(E) > 0$. Let $s_n = n \chi_E$ be simple functions dominated by $f$, then as $n \to \infty$,
\begin{align*}
    \int_X f\,d\mu \geq \int_X s_n\,d\mu = n \mu(E) = \infty,
\end{align*}
which is a contradiction.
\end{proof}

\medskip

\begin{exercise}
Prove that if the functions $f,g:X\to \mathbb{R}$ are
measurable, then the set $\{x \in X \,:\, f(x)=g(x)\}$ is measurable.
\end{exercise}
\begin{proof}
Clearly, $f - g$ is a measurable function, then $(f - g)^{-1}\{0\} = \{x \in X \,:\, f(x)=g(x)\}$ is measurable.
\end{proof}

\medskip

\begin{exercise}
Prove the following stronger version of the
Lebesgue dominated convergence theorem: Suppose that $\{ f_n\}$ is a sequence of complex measurable functions on $X$ such that
$$
f(x)=\lim_{n\to\infty} f_n(x)
$$
for every $x\in X$. Suppose that there is a sequence of functions
$g_n\in L^1(\mu)$ and a function $g\in L^1(\mu)$ such that
$$
\left|f_n(x)\right|\leq g_n(x)
\qquad
\mbox{{\em (}$n=1,2,3,\dots; x\in X${\em )}}
$$
and
$$
\lim_{n\to\infty} \int_X|g_n-g|\, d\mu = 0.
$$
Then $f\in L^1(\mu)$ and
$$
\lim_{n\to\infty} \int_X|f_n-f|\, d\mu = 0.
$$
\end{exercise}
\begin{proof} 
Since $\lim_{n\to\infty} \int_X |g_n - g|\, d\mu = 0$, then by Riesz theorem, there is a subsequence $\{g_{n_k}\}$ such that for $\lim_{k\to\infty} g_{n_k}(x) = g(x)$ a.e. Also, since $f(x) = \lim_{n\to\infty} f_n(x)$ and $\left|f_n(x)\right|\leq g_n(x)$, $f$ is measurable and $\left|f\right| \leq g$. Then, $\left|f_n - f\right| \leq \left|f_n\right| + \left|f\right| \leq g_n + g$, and by Fatou's lemma \ref{theorem_212}, we have
\begin{align*}
    \int_X (g_n + g)\, d\mu & = \int_X \lim_{n\to\infty} (g_n + g - \left|f_n - f\right|)\, d\mu \\
    & = \int_X (g_n + g)\, d\mu + \int_X \liminf_{n\to\infty} (-\left|f_n - f\right|)\, d\mu \\
    & \leq \int_X (g_n + g)\, d\mu + \liminf_{n\to\infty} \int_X (-\left|f_n - f\right|)\, d\mu \\
    & = \int_X (g_n + g)\, d\mu - \limsup_{n\to\infty} \int_X \left|f_n - f\right|\, d\mu,
\end{align*}
also, since $g, g_n \in L^1(\mu)$, $\int_X (g_n + g)\, d\mu < \infty$. Hence,
\begin{align*}
    \limsup_{n\to\infty} \int_X \left|f_n - f\right|\, d\mu \leq 0,
\end{align*}
which implies
\begin{align*}
    \lim_{n\to\infty} \int_X \left|f_n - f\right|\, d\mu = 0.
\end{align*}
\end{proof}

\medskip

\begin{exercise}
For $f:[0,1] \to \mathbb{R}$, let $E \subset \{x \,:\, f'(x)\, \text{exists}\}$. Prove that if $\left|E\right| = 0$, then $\left|f(E)\right| = 0$.
\end{exercise}
\begin{proof}
Suppose $f'(x_0)$ exists and $x_0 \in E$, then $f$ is continuous at a neighborhood of $x_0$. Hence, $E$ can be covered by closed balls $\{\overline{B}(x_i,r_i)\}$ such that $f$ is continuous on each balls $\overline{B}(x_i,r_i)$ and $\sum^\infty_{i=1} r_i < \varepsilon/L $, where
\begin{align*}
    L = \max \left\{f'(x) \,:\, x \in \bigcup^\infty_{i=1} \overline{B}(x_i,r_i)\right\}.
\end{align*}
Similar to Lipschitz continuity, we have
\begin{align*}
    f(E) \subset \bigcup^\infty_{i=1} f(\overline{B}(x_i,r_i)) \subset \bigcup^\infty_{i=1} \overline{B}(f(x_i),Lr_i) , \quad \sum^\infty_{i=1} Lr_i < \varepsilon.
\end{align*}
Hence, $f(E)$ can be covered by balls of radius $Lr_i$ with $\sum^\infty_{i=1} Lr_i < \varepsilon$, and since it holds for all $\varepsilon$, $\mu(f(E)) = 0$.
\end{proof}

\medskip

\medskip

\begin{exercise}
Show that for any Lebesgue measurable set $E \subset \mathbb{R}$ with $\left|E\right| = 1$ there is a Lebesgue measurable subset $A \subset E$ such that $\left|A\right| = 1/2$.
\end{exercise}
\begin{proof}
Define $f: [0,\infty) \to \mathbb{R}$ by $f(x) = \mu(E \cap [-x,x])$. Clearly, $f$ is well-defined since $[-x,x]$ is Lebesgue measurable for any $x$ and so is $E \cap [-x,x]$. We claim that $f$ is continuous. Indeed, for any $x, y \in \mathbb{R}$, assuming that $x \geq y$, we have
\begin{align*}
    \left|f(x) - f(y)\right| & = \left|\mu(E \cap [-x,x]) - \mu(E \cap [-y,y])\right| \\
    & = \left|\mu(E \cap [-x,x] \setminus [-y,y])\right| \\
    & = \left|\mu(E \cap ([-x,-y) \setminus (y,x]))\right| \\
    & \leq \left|\mu([-x,-y) \setminus (y,x])\right| \\
    & = \left|\mu([-x,-y)) + \mu((y,x])\right| = 2 \left|x - y\right|,
\end{align*}
which implies that $f$ is Lipschitz and hence continuous.

Now $f(0) = \mu(E \cap \{0\}) = 0$ and $f(\infty) = \mu(E) = 1$, and by intermediate value theorem, there is $c \in (0, \infty)$ such that $f(c) = 1/2$. Let $A = E \cap [-c,c]$, and clearly, $\mu(A) = f(c) = 1/2$.
\end{proof}

\medskip

\begin{exercise}
Suppose that $f_n:X \to [0,\infty]$ is a sequence of measurable functions such that $f_1\in L^1(\mu)$ and
$f_1 \geq f_2 \geq \cdots\geq 0$, $f_n(x) \to f(x)$ for every $x\in X$. Prove that
$$
\lim_{n\to\infty} \int_X f_n(x)\, d\mu = \int_X f(x)\, d\mu.
$$
Show also that the assumption $f_1\in L^1(\mu)$ cannot be removed.
\end{exercise}
\begin{proof}
This is obvious by Lebesgue dominated convergence theorem. Also, the condition $f_1\in L^1(\mu)$ cannot be removed, since otherwise, $f$ may not be integral functions.
\end{proof}

\medskip

\begin{definition}
We say that a function $\phi:\mathbb{R}\to\mathbb{R}$ is Cauchy if $\phi(x+y) = \phi(x) + \phi(y)$, for all $x,y\in\mathbb{R}$. Clearly linear functions $\phi(x) = ax, a \in \mathbb{R}$ are Cauchy.
\end{definition}

\begin{exercise}\label{exercise_529}
Prove that if a Cauchy function is continuous, then it is linear.
\end{exercise}
\begin{proof}
Suppose $\phi$ is Cauchy and continuous, then it suffices to show that $\phi(x) = ax$ for all $x$. 
\begin{enumerate}[label=(\alph*)]
    \item If $q = 0$, $\phi(0) = \phi(0 + 0) = 2 \phi(0)$ implies that $\phi(0) = 0$.
    
    \item If $q \in \mathbb{Q}, q > 0$. Since $f(\alpha x) = \alpha f(x)$, for $\alpha \in \mathbb{N}, x \in \mathbb{Q}$, substituting $x$ by $x/\alpha$ and multiplying both sides by $\beta/\alpha$, where $\beta \in \mathbb{N}$ yields
    \begin{align*}
        \frac{\beta}{\alpha} f(x) = \beta f\left(\frac{x}{\alpha}\right), \alpha,\beta \in \mathbb{N}, x \in \mathbb{Q}.
    \end{align*}
    Since $f$ is Cauchy, then for the left hand side, \begin{align*}
        \frac{\beta}{\alpha} f(x) = \frac{\beta}{\alpha} \cdot \alpha f\left(\frac{x}{\alpha}\right) = \beta f\left(\frac{x}{\alpha}\right) = f\left(\frac{\beta}{\alpha} x\right),
    \end{align*}
    and similarly, for the right hand side, we have
    \begin{align*}
        \beta f\left(\frac{x}{\alpha}\right) = \frac{\beta}{\alpha} f(x).
    \end{align*}
    Therefore,
    \begin{align*}
        f\left(\frac{\beta}{\alpha} x\right) = \frac{\beta}{\alpha} f(x).
    \end{align*}
    Letting $q = \beta/\alpha$ gives $f(qx) = qf(x)$, $q, x \in \mathbb{Q}, q > 0$, which implies $f(q) = qf(1) = cq$. Hence, $f$ is Cauchy for rational numbers.
    
    \item If $q \in \mathbb{Q}, q < 0$. Since $\phi(x+(-x)) = \phi(0) = 0$, we have $\phi(x) = - \phi(-x)$. Then, $f(q) = - f(-q) = -c(-q) = cq, c = f(1)$ for $q \in \mathbb{Q}, q < 0$.
\end{enumerate}

Now we proved that $f: \mathbb{Q} \to \mathbb{Q}$ is linear. Since $\mathbb{Q}$ is dense in $\mathbb{R}$, for any irrational number $r \in \mathbb{R} \setminus \mathbb{Q}$, there is a sequence $\{q_n\} \subset \mathbb{Q}$ such that $\lim_{n\to\infty} q_n = r$. By continuity of $f$,
\begin{align*}
    \lim_{n\to\infty} \phi(q_n) = \phi(r) = \lim_{n\to\infty} q_n \phi(1) = r \phi(1),
\end{align*}
and hence $\phi(r) = cr$, where $c = \phi(1)$. Thus, $f: \mathbb{R} \to \mathbb{R}$ is linear.
\end{proof}

\medskip

\begin{exercise}\label{exercise_5210}
Prove that if a Cauchy function is continuous at $0$, then it is linear.
\end{exercise}
\begin{proof}
By Problem \ref{exercise_529}, we have $f: \mathbb{Q} \to \mathbb{Q}$ is linear. For any $r \in \mathbb{R} \setminus \mathbb{Q}$, there is a sequence $\{q_n\} \subset \mathbb{Q}$ such that $\lim_{n\to\infty} q_n = - r$. Clearly, $\lim_{n\to\infty} -q_n = - r$. Then, by continuity of $f$ at $0$, we have
\begin{align*}
    \lim_{n\to\infty} \phi(r - q_n) = \phi(0) = 0.
\end{align*}
On the other hand, since $f$ is Cauchy and $q_n \in \mathbb{Q}$,
\begin{align*}
    \lim_{n\to\infty} \phi(r - q_n) = \phi(r) + \lim_{n\to\infty} \phi(-q_n) = \phi(r) + \lim_{n\to\infty} -q_n \phi(1) = \phi(r) - r \phi(1),
\end{align*}
hence, $\phi(r) - r \phi(1) = 0$, and thus, $\phi(r) = r \phi(1)$, which prove that $f$ is linear on $\mathbb{R}$.
\end{proof}

\medskip

\begin{exercise}
Prove that there is a Cauchy function such that $\phi(1)=0$ and $\phi(\pi)=1$. ({\bf Hint:} Clearly, this functions cannot be linear. To prove existence of $\phi$ regard $\mathbb{R}$ as a linear space over the field of rational numbers $\mathbb{Q}$ and use the Hamel basis.)
\end{exercise}
\begin{proof}
Consider $\mathbb{R}$ as a vector space over $\mathbb{Q}$ and $\{x_i\}_{i \in I}$ is the Hamel basis for $\mathbb{R}$. Let $x_1 = 1$ and $x_2 = \pi$, then $x_1$ and $x_2$ are linearly independent in $\mathbb{Q}$, since $\pi$ is irrational.\footnote{Any linearly independent subset of a vector space can be extended into a basis.} Since any $x \in \mathbb{R}$ can be written as $x = \sum_{i\in I} \lambda_i x_i$, and $f: \mathbb{R} \to \mathbb{R}$ is additive, $f(x)$ is well defined for all $x \in \mathbb{R}$ and is given by
\begin{align*}
    f(x) = f\left(\sum_{i\in I} \lambda_i x_i\right) = \sum_{i \in I} f(\lambda_i x_i) = \sum_{i \in I} \lambda_i f(x_i)
\end{align*}
Then any function, which is defined on a basis can be uniquely extended into a function in the space. Now we can let $\phi(1) = 0$,$\phi(\pi) = 1$ and $\phi(x), x \neq 1,\pi$ be whatever.
\end{proof}

\medskip

\begin{exercise}
Prove that is a Cauchy function is not linear, then it is not Lebesgue measurable. ({\bf Hint:} Use the Steinhaus theorem or use the Lusin theorem.)
\end{exercise}
\begin{proof}
Suppose that $f$ is measurable. For any interval $I$ of length $\varepsilon$, we have $E \coloneqq f^{-1}(I)$ is measurable. Then, we have
\begin{align*}
    f^{-1}((-\varepsilon,\varepsilon)) = f^{-1}(I - I) = E - E.
\end{align*}
By the Steinhaus theorem, $E - E$ contains a neighborhood of $0$. Thus, for some $\delta > 0$, if $\left|x\right| < \delta$, then $x \in E - E$ and hence $\left|f(x)\right| < \varepsilon$, which implies $f$ is continuous at $0$. By Exercise \ref{exercise_5210}, this is a contradiction.
\end{proof}

\medskip

\begin{exercise}\label{exercise_5213}
Prove that if $f_n$ converges in measure to $f$ and converges in measure to $g$, then $f = g$ a.e.
\end{exercise}
\begin{proof}
Let $E = \{x \,:\, f(x) \neq g(x)\}$. It suffices to prove that $\mu(E) = 0$. 

Let $\varepsilon > 0$, since $f_n \xrightarrow[]{\mu} f$ and $f_n \xrightarrow[]{\mu} g$, then $f,g,f_n$ are measurable. Convergence in measure implies that
\begin{align*}
    \lim_{n\to\infty} \mu\left(\left\{x \,:\, \left|f_n(x) - f(x)\right| \geq \frac{\varepsilon}{2}\right\}\right) & = 0, \\
    \lim_{n\to\infty} \mu\left(\left\{x \,:\, \left|f_n(x) - g(x)\right| \geq \frac{\varepsilon}{2}\right\}\right) & = 0.
\end{align*}
Also, for some $x \in X$, if $\left|f(x) - g(x)\right| \geq \varepsilon$, then $x$ satisfies either $\left|f_n(x) - g(x)\right| \geq \varepsilon/2$ or $\left|f_n(x) - f(x)\right| \geq \varepsilon/2$. Indeed, 
\begin{align*}
    \varepsilon \leq \left|f(x) - g(x)\right| \leq \left|f_n(x) - g(x)\right| + \left|f_n(x) - f(x)\right|.
\end{align*}
Thus,
\begin{align*}
    \mu\left(\left\{x \,:\, \left|f(x) - g(x)\right| \geq \varepsilon\right\}\right) & \leq \mu\left(\left\{x \,:\, \left|f_n(x) - f(x)\right| \geq \frac{\varepsilon}{2}\right\}\right) \\
    & + \mu\left(\left\{x \,:\, \left|f_n(x) - g(x)\right| \geq \frac{\varepsilon}{2}\right\}\right) \xrightarrow[]{n\to\infty} 0,
\end{align*}
which implies $f = g$ a.e.
\end{proof}

\begin{proof}[Second Proof of Exercise \ref{exercise_5213}]
By Riesz theorem \ref{theorem_225}, there are subsequence $\{f_{n_k}\}$ such that $f_{n_k} \xrightarrow[]{\mu} f$ a.e. and another subsequence $\{f_{n_l}\}$ such that $f_{n_l} \xrightarrow[]{\mu} g$ a.e. This means $f_{n_k}(x) \to f(x)$ for $x \in X \setminus N_1$, where $\mu(N_1) = 0$. Similarly, $f_{n_l}(x) \to g(x)$ for $x \in X \setminus N_2$, where $\mu(N_2) = 0$.

Let $N = N_1 \cup N_2$ and clearly, $\mu(N) = 0$. Also, for all $x \in X \setminus N$, we have $f_{n_k}(x) \to f(x)$ and $f_{n_l}(x) \to g(x)$, so $f(x) = g(x)$ for all $x \in X \setminus N, \mu(N) = 0$. Thus, $f = g$ a.e.
\end{proof}

\medskip

\begin{exercise}
Show an example of a sequence of measurable functions $f_n:[0,1]\to\mathbb{R}$ that converge to the function $f=0$ in measure but not a.e.
\end{exercise}
\begin{proof}
Define a sequence $\{E_n\}$ of sets as follows:\footnote{This construction is based on an counterexample in Wikipedia \cite{34}.}
\begin{align*}
    E_1 & = \left[0,1\right], \\ 
    E_2 & = \left[0,\frac{1}{2}\right], \, E_3 = \left[\frac{1}{2},1\right] \\
    E_4 & = \left[0,\frac{1}{4}\right], \, E_5 = \left[\frac{1}{4},\frac{1}{2}\right],\, E_6 = \left[\frac{1}{2},\frac{3}{4}\right], \, E_7 = \left[\frac{3}{4},1\right] \\
    & \cdots
\end{align*}
Then we can define the sequence $\{f_n\}$ of functions by $f_n(x) = \chi_{E_n}$. Hence, $f_n \xrightarrow[]{\mu} f = 0$ but does not converge to $0$ a.e. 

Indeed, $f_n$ differs from $f = 0$ on an interval of length $1/2^k$ for some  $k \in \mathbb{N}$. Then, for any $\varepsilon > 0$, we have
\begin{align*}
    \lim_{n\to\infty} \mu\left(\left\{x \,:\, \left|f_n(x) - f(x)\right| \geq \varepsilon \right\}\right) \leq \lim_{k\to\infty} \frac{1}{2^k} = 0,
\end{align*}
which implies $f_n \xrightarrow[]{\mu} 0$. However, for any $x \in [0,1]$, there are infinitely many $f_n$ such that $f_n(x) = 1$, and hence $f_n$ does not converge pointwise to $0$. Thus, $f_n$ does not converge to $f$ a.e.
\end{proof}

\medskip

\begin{exercise}
Show an example of a sequence $f_n: \mathbb{R} \to \mathbb{R}$ of measurable functions that converge of a function $f$ a.e. but not in measure.
\end{exercise}
\begin{proof}
Define the sequence $\{f_n\}$ of functions $f_n: \mathbb{R} \to \mathbb{R}$ by $f_n = \chi_{[n,n+1]}$ for all $n \in \mathbb{N}$. Then, $f_n$ converges to $f = 0$ almost everywhere, but it does not converges in measure.

Indeed, for any $x \in \mathbb{R}$, there is $N > 0$ such that $x \leq N$ which implies $f_n(x) = f = 0$ for all $n \geq N$. However, 
\begin{align*}
    \lim_{n\to\infty} \mu\left(\left\{x \,:\, \left|f_n(x) - f(x)\right| \geq \varepsilon \right\}\right) = 1,
\end{align*}
which implies $f_n$ does not converge to $f = 0$ in measure.
\end{proof}

\medskip

\begin{definition}
Let $(X,\mu)$ be a complete measure space.
For a function $f : X \to [0,\infty]$ defined $\mu$-a.e. on $X$, the upper integral is defined by
$$\int^*_X f \,d\mu = \inf \int_X \phi \,d\mu,$$
where the infimum is taken over all $\mu$-measurable functions $\phi$ satisfying $0 \leq f(x) \leq \phi(x)$ for $\mu$-a.e. $x \in X$.
We do note require $f$ to be measurable. Clearly, for measurable functions, the upper integral coincides with the Lebesgue one.
\end{definition}

\medskip

\begin{exercise}
Prove that if $\int_X^* f \,d\mu=0$, then $f = 0$ $\mu$-a.e. and hence it is measurable.
\end{exercise}
\begin{proof}
Since
\begin{align*}
    \int^*_X f \,d\mu = \inf \int_X \phi \,d\mu = 0,
\end{align*}
then for any $n \in \mathbb{N}$, there is a measurable function $\phi_n$ such that $\int_X \phi_n \,d\mu < 1/n$ and $0 \leq f(x) \leq \phi_n(x)$ for almost every $x \in X$. 

Now we define $g_n(x) = \inf_{k\leq n} \phi_k(x)$. Clearly, $0 \leq f(x) \leq g_n(x)$ for almost every $x \in X$. Also, we have $\int_X g_n \,d\mu < 1/n$. Since $\{g_n\}$ is bounded below and decreasing, the sequence has a limit, denoted by $g$. By Fatou's lemma,\footnote{Also, Lebesgue dominated convergence theorem could work here.}we have
\begin{align*}
    \int_X g \,d\mu = \int_X \left(\liminf_{n\to\infty} g_n\right) \,d\mu \leq \liminf_{n\to\infty} \int_X g_n \,d\mu \leq \liminf_{n\to\infty} \frac{1}{n} = 0,
\end{align*}
which implies $g = 0$ a.e. Since $0 \leq f \leq g$, $f = 0$ for almost every $x \in X$.
\end{proof}

\medskip

\begin{exercise}[{\bf Nonmeasurable monotone convergence theorem}]
Let $f_n: X \to [0,\infty]$ be a monotone sequence of (not necessarily measurable) functions, i.e.
$0 \leq f_1(x) \leq f_2(x) \leq \ldots$ for $\mu$-a.e. $x \in X$. Prove that if $f(x):=\lim_{n \to \infty} f_n(x)$, then
\begin{equation}
    \lim_{n\to \infty} \int_X^* f_n \,d\mu = \int_X^* f \, d\mu.
\end{equation}
\end{exercise}
\begin{proof}
~\begin{enumerate}[label=(\alph*)]
    \item If $\lim_{n\to\infty} f_n(x) = \infty$, the equality holds naturally.
    
    \item If $\lim_{n\to\infty} f_n(x) < \infty$, firstly, since $f_n(x) \leq f(x)$ for almost every $x \in X$, we have
    \begin{align*}
        \lim_{n\to \infty} \int_X^* f_n \,d\mu \leq \int_X^* f \,d\mu.
    \end{align*}
    It remains to prove that
    \begin{align}\label{exercise_5217_equ1}
        \lim_{n\to \infty} \int_X^* f_n \,d\mu \geq \int_X^* f \,d\mu.
    \end{align}
    For any $n \in \mathbb{N}$, there exists a measurable function $\phi_n$ such that $0 \leq f_n(x) \leq \phi_n(x)$ for almost every $x \in X$ and 
    \begin{align*}
        \int_X^* f_n \,d\mu \leq \int_X^* \phi_n \,d\mu \leq \int_X^* f_n \,d\mu + \frac{1}{n}.
    \end{align*}
    Now define $\varphi_n = \inf_{k\geq n} \phi_k$, clearly $\varphi_n$ is measurable. Also, $\{\varphi_n\}$ is an increasing sequence and bounded above, and hence it has a limit, denoted by $\varphi$, that is $\varphi = \lim_{n\to\infty} \varphi_n$. Note that
    \begin{align*}
        0 \leq f = \lim_{n\to\infty} f_n \leq \lim_{n\to\infty} \phi_n \leq \varphi,
    \end{align*}
    then by monotone convergence theorem, we have
    \begin{align*}
        \int_X^* f \,d\mu \leq \int_X^* \varphi \,d\mu = \lim_{n\to\infty} \int_X^* \varphi_n \,d\mu \leq \lim_{n\to\infty} \int_X^* \phi_n \,d\mu \leq \lim_{n\to\infty} \left(\int_X^* f_n \,d\mu + \frac{1}{n}\right).
    \end{align*}
    As $n \to \infty$, \eqref{exercise_5217_equ1} holds.
\end{enumerate}
\end{proof}

\medskip

\begin{definition}
We say $\phi:X \to [0,\infty]$ is a \textit{step function} if it is $\mu$-measurable and attains at most countably many values (we allow infinite values). That is, $\phi$ is a step function if there exist  disjoint $\mu$-measurable subsets $A_i \subset X $ and $0 \leq a_i \leq \infty $ such that
$$
\phi(x) = \sum_{i=1}^\infty a_i \chi_{A_i}(x).
$$
Note that the difference with the definition of a simple function is that a simple function is finite and has only finitely many values.
\end{definition}

\medskip

\begin{exercise}
Let $f:X \to [0,\infty]$ be any function. Prove that
$$
\int^*_X f \,d\mu = \inf \int_X \varphi \,d\mu,
$$
where the infimum is over all step functions $\varphi$ satisfying
$0 \leq f(x) \leq \varphi(x)$ for all $x\in X$.
\end{exercise}
\begin{proof}
By the definition of upper integral, there exists a measurable function $\phi: X \to [0,\infty]$ such that $\int^*_X f \,d\mu = \int_X \phi \,d\mu$. It suffices to show that $\int_X \phi \,d\mu = \inf \int_X \varphi \,d\mu$ where the infimum is taken over all step functions $\varphi$ such that $0 \leq f(x) \leq \varphi(x)$ for all $x \in X$.
\begin{enumerate}[label=(\alph*)]
    \item For $\int^*_X f \,d\mu = \infty$, the equality holds naturally.
    
    \item For $\int^*_X f \,d\mu < \infty$. By a proposition in Gustav Holzegel's {\em Measure and Integral} \cite{35}: Let $s = \sum^n_{i=1} \alpha_i \chi_{A_i}$ be a simple function with $\mu(A_i) < \infty$. For any $\varepsilon > 0$, there is a step function $\varphi$ such that 
    \begin{align*}
        \mu\left(\left\{x \,:\, \varphi(x) \neq s(x)\right\}\right) < \varepsilon.
    \end{align*}
    Thus, any simple function can be approximated by a step function, and any measurable function can be approximated by a simple function, then we are done.
\end{enumerate}
\end{proof}

\medskip

\begin{exercise}
Prove that if  $f,f_k \in L^1(\mu)$, $f_k \to f$ a.e. and $\left\|f_k\right\|_1 \to \left\|f\right\|_1$, then $\left\|f_k - f\right\|_1 \to 0$. ({\bf Hint:} Mimic the proof of the dominated convergence theorem.)
\end{exercise}
\begin{proof}
Since $\left|f - f_k\right| \leq \left|f\right| + \left|f_k\right|$, we have $\left|f\right| + \left|f_k\right| - \left|f - f_k\right| \geq 0$. By Fatou's lemma, we have
\begin{align*}
    0 \leq \int_X \liminf_{k\to\infty} (\left|f\right| + \left|f_k\right| - \left|f - f_k\right|) \,d\mu \leq \liminf_{k\to\infty} \int_X (\left|f\right| + \left|f_k\right| - \left|f - f_k\right|) \,d\mu,
\end{align*}
and since $f_k \to f$ a.e. and $\left\|f_k\right\|_1 \to \left\|f\right\|_1$, we have
\begin{align*}
    0 \leq \int_X 2 \left|f\right| \,d\mu & \leq \int_X \left|f\right|\,d\mu + \liminf_{k\to\infty} \int_X \left|f_k\right|\,d\mu + \liminf_{k\to\infty} \int_X -\left|f - f_k\right|\,d\mu \\
    & \leq \int_X 2 \left|f\right|\,d\mu - \limsup_{k\to\infty} \int_X \left|f - f_k\right|\,d\mu.
\end{align*}
Since $\int_X 2 \left|f\right|\,d\mu$ is finite, we have
\begin{align*}
    \limsup_{k\to\infty} \int_X \left|f - f_k\right|\,d\mu \leq 0,
\end{align*}
which implies
\begin{align*}
    \lim_{k\to\infty} \int_X \left|f - f_k\right|\,d\mu = 0.
\end{align*}
\end{proof}

\medskip

\begin{exercise}
Prove that if
\begin{align*}
    \int_{\mathbb{R}^n} \left|\log \left|f_k\right|\right| \,d\mu \xrightarrow[]{k \to \infty} 0,
\end{align*}
then there is a subsequence $\{f_{k_i}\}$ such that $\left|f_{k_i}\right| \to 1$ a.e.
\end{exercise}
\begin{proof}
Clearly, $\log \left|f_k\right| \xrightarrow[]{\mu} 0$. By Riesz theorem, there is a subsequence $\{\log \left|f_{n_k}\right|\}^\infty_{k=1}$ such that $\log \left|f_{n_k}\right| \xrightarrow[]{\mu} 0$ a.e. Then,
\begin{align*}
    \lim_{k\to\infty} \left|f_{n_k}\right| = \lim_{k\to\infty} e^{\log \left|f_{n_k}\right|} = 1,
\end{align*}
for almost every $x \in \mathbb{R}^n$.
\end{proof}

\medskip

\begin{exercise}
Prove that if $f: \mathbb{R} \to \mathbb{R}$ is continuous, then
\begin{align*}
    \left\|f\right\|_2 \leq \liminf_{m \to \infty} \left(\frac{1}{m} \sum^\infty_{k = -\infty}  f\left(\frac{k}{m}\right)^2 \right)^{1/2}.
\end{align*}
({\bf Hint:} Use Fatou's lemma.)
\end{exercise}
\begin{proof}
Define $f_m: \mathbb{R} \to \mathbb{R}$ by 
\begin{align*}
    f_m(x) = f\left(\frac{\floor*{xm}}{m}\right), \quad x \in \left[\frac{\floor*{xm}}{m}, \frac{\floor*{xm}+1}{m}\right),
\end{align*}
where $\floor*{x}$ denote the greatest integer that less than $x$. Then, $f_m$ is a step function that converges pointwise to $f$, since $f$ is continuous. Thus, by Fatou's lemma, we have
\begin{align*}
    \left\|f\right\|_2^2 = \int_X \left(\liminf_{m\to\infty} \left|f_m\right|^2 \right) \,d\mu \leq \liminf_{m\to\infty} \int_X \left|f_m\right|^2 \,d\mu = \liminf_{m\to\infty} \frac{1}{m} \sum^\infty_{k=-\infty} f\left(\frac{k}{m}\right)^2,
\end{align*}
which implies the inequality above.
\end{proof}

\medskip

\begin{exercise}\label{exercise_7222}
Suppose that $f_n\to f$ in measure and $g_n\to g$ in measure. Show that $f_ng_n\to fg$ in measure if $\mu(X)<\infty$. Show that the claim is not necessarily true if $\mu(X)=\infty$.
\end{exercise}
\begin{proof}
Since $\mu(X) < \infty$, note that the sets $G_m \coloneqq \{x \,:\, \left|f\right| > m\}$ are decreasing, then $\bigcap G_m = \{x \,:\, \left|f\right|  = \infty\}$ is a set of measure zero. Hence, there exists $M > 0$ such that $\mu(\{x \,:\, \left|f_n(x)\right| > M \}) < \varepsilon$ for all $n$ and $\mu(\{x \,:\, \left|g(x)\right| > M \}) < \varepsilon$. Now by $\left|f_ng_n - fg\right| \leq \left|f_n(g_n - g)\right| + \left|(f_n - f)g\right|$, we have
\begin{align*}
    \{\left|f_ng_n - fg\right| > \varepsilon \} \subset \left\{\left|f_n\right| \left|g_n - g\right| > \frac{\varepsilon}{2} \right\} \cup \left\{\left|g\right| \left|f_n - f\right| > \frac{\varepsilon}{2} \right\}.
\end{align*}
Now observe that
\begin{align*}
    \left\{\left|f_n\right| \left|g_n - g\right| > \frac{\varepsilon}{2} \right\} \subset \{\left|f_n(x)\right| > M \} \cup \left\{\left|g_n - g\right| > \frac{\varepsilon}{2M} \right\},
\end{align*}
and
\begin{align*}
    \left\{\left|g\right| \left|f_n - f\right| > \frac{\varepsilon}{2} \right\} \subset \{\left|g\right| > M \} \cup \left\{\left|f_n - f\right| > \frac{\varepsilon}{2M} \right\}.
\end{align*}
Thus, we have
\begin{align*}
    \mu\left(\{\left|(f_ng_n)(x) - (fg)(x)\right| > \varepsilon \} \right) & \leq \mu(\{\left|f_n(x)\right| > M \}) + \mu\left(\left\{\left|g_n(x) - g(x)\right| > \frac{\varepsilon}{2M}\right\}\right) \\
    & + \mu\left(\left\{\left|f_n(x) - f(x)\right| > \frac{\varepsilon}{2M}\right\}\right) + \leq \mu(\{\left|g(x)\right| > M \}) < 4 \varepsilon,
\end{align*}
which implies that $f_ng_n \to fg$ in measure.

For a counterexample when $\mu(X) = \infty$. Take $X = \mathbb{R}$, $\mu$ be the Lebesgue measure and $f = g = x^2$ unbounded. Let $f_n = g_n = f + 1/n$. Then for $x \in [n,\infty)$, we have 
\begin{align*}
    \left|f_n(x)g_n(x) - f(x)g(x)\right| \geq \frac{2x^2}{n} \geq 2n,
\end{align*}
and hence as $n \to \infty$,
\begin{align*}
    \mu(\{\left|f_n(x)g_n(x) - f(x)g(x)\right| \geq 1\}) \geq \mu([n,\infty)) \not\to 0.
\end{align*}
Thus, $f_ng_n \not\to fg$ in measure.
\end{proof}

\medskip

\begin{proof}[Second Proof of Exercise \ref{exercise_7222}]
Choose any increasing sequence $\{n_k\}$ of integers. Since $f_n \to f$ in measure, the same is true for $f_{n_k}$. By Riesz theorem, there is a subsequence $\left\{n_{k_m}\right\}$ such that $f_{n_{k_m}} \to f$ $\mu$-a.e. Then, since $g_n \to g$ in measure, so does $\left\{g_{n_{k_m}}\right\}$. Again, by Riesz theorem, there is a subsequence $\left\{n_{k_{m_j}}\right\}$ such that $g_{n_{k_{m_j}}} \to g$ $\mu$-a.e. Since $f_{n_{k_m}} \to f$ $\mu$-a.e., then the same is true for $\left\{f_{n_{k_{m_j}}}\right\}$.

Now, $\left\{f_{n_{k_{m_j}}}\right\}$ and $\left\{g_{n_{k_{m_j}}}\right\}$ converge to $f$ and $g$ $\mu$-a.e., then $f_{n_{k_{m_j}}} g_{n_{k_{m_j}}} \to fg$ $\mu$-a.e. Since $\mu(X) < \infty$, then we have $f_{n_{k_{m_j}}} g_{n_{k_{m_j}}} \to fg$ in measure as well. And this implies $f_ng_n \to fg$ in measure.
\end{proof}

\medskip

\section{$L^p$-space}

\begin{exercise}
Let $\tau_y f(x)=f(x+y)$. Prove that if $1\leq p<\infty$ and $f\in L^p(\mathbb{R}^n)$, then
\begin{align*}
    \lim_{\left|y\right| \to \infty} \left\|\tau_y f + f\right\|_p = 2^{1/p} \left\|f\right\|_p.
\end{align*}
\end{exercise}
\begin{proof}

\end{proof}

\medskip

\begin{exercise}
Prove that if $0 < \mu(X) < \infty$, then
\begin{align*}
    \left(\avint_X \left|f\right|^p \,d\mu \right)^{1/p} \leq \left(\avint_X \left|f\right|^q \,d\mu \right)^{1/q},
\end{align*}
for all $0 < p < q < \infty$.
\end{exercise}
\begin{proof}
Recall Jensen's inequality:
\begin{align*}
    \varphi\left(\frac{1}{\mu(X)} \int_X f\,d\mu \right) \leq \frac{1}{\mu(X)} \int_X \varphi \circ f\,d\mu,
\end{align*}
where $\varphi$ is convex function. Take $\varphi = x^{q/p}$, since $p < q$, then $\varphi$ is convex. Hence,
\begin{align*}
    \left(\frac{1}{\mu(X)} \int_X \left|f\right|^p\,d\mu \right)^{q/p} \leq \frac{1}{\mu(X)} \int_X \left|f\right|^q\,d\mu,
\end{align*}
and taking $q$th root implies the result:
\begin{align*}
    \left(\frac{1}{\mu(X)} \int_X \left|f\right|^p\,d\mu \right)^{1/p} \leq \left(\frac{1}{\mu(X)} \int_X \left|f\right|^q\,d\mu\right)^{1/q}.
\end{align*}
\end{proof}

\medskip


\section{Integration on Product Spaces}

\medskip

\begin{exercise}[Proposition 3.7 in \cite{38}]
Prove that if $X$ is a compact metric space, then the metric space of continuous functions $C(X)$ is separable. ({\bf Hint:} {\em Use the Stone-Weierstrass theorem and distance functions $y\mapsto d(y_0,y)$.})
\end{exercise}
\begin{proof} 
When $X$ consists of a singleton, then $C(X)$ is equal to $\mathbb{R}$ and hence separable. Now we assume $X$ has more than one points. Since $X$ is compact, then there is a sequence of ball $\{B_i\}$ whose centers $\{x_i\}$ form a dense set in $X$.\footnote{Indeed, this comes from Proposition 2.11 in \cite{39}: A closed set has the finite cover property if and only if it has the
finite intersection property.
\begin{enumerate}[label=(\alph*)]
    \item A set $E \subset X$ satisfies the {\bf finite cover property} if whenever $\{G_\alpha\}$ is an open cover of $E$, there is a subcollection consisting of finitely many $G_{\alpha_1}, \cdots, G_{\alpha_K}$ such that $E \subset \bigcup^K_{i=1} G_{\alpha_i}$.
    
    \item A set $E$ satisfies the {\bf finite intersection property} if whenever $\{F_\alpha\}$ are relatively closed set in $E$ satisfying $\bigcap^K_{i=1} F_{\alpha_i} \neq \varnothing$ for any finite subcollection $\{F_{\alpha_i}\}$. 
\end{enumerate}
}Define $f_i(x) = d(x,x_i)$ and let $\mathcal{M} \subset C(X)$ consist of functions which are finite product of $f_i$'s. Now let
\begin{align*}
    \mathcal{A} = \left\{\sum^N_{k=1} a_kh_k \,:\, h_k \in \mathcal{M}, a_j \in \mathbb{Q}\right\}.
\end{align*}
It is easy to check that $\mathcal{A}$ is an algebra in $C(X)$. To verify separating points property, let $y_1$ and $y_2$ be two distinct points in $X$. The function $f(y) = d(y,x_1)$ satisfies $f(y_1) = 0$ and $f(y_2) \neq 0$. By density of $\{x_i\}$, we can find some $x_i$ close to $y_1$ such that the function $f_i(x) = d(x,x_i)$ separates $y_1$ and $y_2$. On the other hand, given any point $x_0$ we can fix another distinct point $y_0$ so that the function $d(x,y_0)$ is nonvanishing at $x_0$. By density of $\{x_i\}$ again, there is some $x_j$ close to $y_0$ such that $f_j(x_0) \neq 0$.  By the Stone-Weierstrass theorem, $\mathcal{A}$ is dense in $C(X)$.

It remains to prove that $\mathcal{A}$ is countable. Let $\mathcal{A}_n$ be the subset of $\mathcal{A}$ which only contains finitely many functions $f_1, \cdots, f_n$. We have $\mathcal{A} = \bigcup_n \mathcal{A}_n$, so it suffices to prove that each $\mathcal{A}_n$ is countable. Each function in $\mathcal{A}_n$ is composed of finitely many terms of the form $f_{n_1}^{a_1} \cdots f_{n_k}^{a_k}, n_j \in \{1,\cdots,n\}$. Let $\mathcal{A}_n^m \subset \mathcal{A}_n$ consists of all those functions whose degree is less than or equal to $m$. It is clear $\mathcal{A}_n^m$ is countable, so is $\mathcal{A}_n = \bigcup_m \mathcal{A}_n^m$.
\end{proof}

\medskip

\begin{exercise}
Prove that if $f:\mathbb{R}^n\to [0,\infty)$ is Lebesgue measurable, then the set
$$
U_f = \{(x,y) \in \mathbb{R}^n \times \mathbb{R} \,:\, x \in \mathbb{R}^n, 0 \leq y\leq f(x)\} \subset \mathbb{R}^{n+1}
$$
is Lebesgue measurable and
$$
\mathcal{L}_{n+1}(U_f)=\int_{\mathbb{R}^n} f \,d\mathcal{L}_n.
$$
\end{exercise}
\begin{proof}
Let $s: \mathbb{R}^n \to [0,\infty]$ be a simple function, defined as
\begin{align*}
    s(x) = \sum^n_{i=1} \alpha_i \chi_{A_i}(x).
\end{align*}
Clearly, the set 
\begin{align*}
    U_s = \bigcup^n_{i=1} A_i \times [0,\alpha_i]
\end{align*}
is measurable. Let $\{s_n\}$ be a sequence of nonnegative simple functions that converges to $f$ pointwisely. Since each $U_{s_n}$ is measurable, then the set
\begin{align*}
    U_f = \bigcup^\infty_{n=1} U_{s_n}
\end{align*}
is also measurable.

Observe that
\begin{align*}
    \int_{\mathbb{R}^{n+1}} \chi_{U_f}\,d\mathcal{L}_{n+1} = \mathcal{L}_{n+1} (U_f),
\end{align*}
and $\chi_{U_f}(x,y) = \chi_{[0,f(x)]}(y)$. By the Fubini theorem, we have
\begin{align*}
    \mathcal{L}_{n+1} (U_f) & = \int_{\mathbb{R}^{n+1}} \chi_{U_f}\,d\mathcal{L}_{n+1} = \int_{\mathbb{R}^n} \left(\int_{\mathbb{R}} \chi_{[0,f(x)]}(y) \,d\mathcal{L}\right) \,d\mathcal{L}_n = \int_{\mathbb{R}^n} f(x) \,d\mathcal{L}_n.
\end{align*}
\end{proof}

\medskip

\begin{exercise}
Show an example that the Fubini theorem is not true if the measures are not $\sigma$-finite.
\end{exercise}
\begin{proof}\cite{40}
Let $\left([0,1], \mathfrak{B}([0,1]), \mu\right)$ and $\left([0,1], \mathcal{P}([0,1]), \lambda\right)$ be measurable spaces with the Lebesgue measure $\mu$ and the counting measure $\lambda$ respectively, and the counting measure is not $\sigma$-finite.

Consider the set $D = \left\{(x,x) \,:\, x \in [0,1] \right\}$, which is a closed subset of $[0,1] \times [0,1]$. Hence, 
\begin{align*}
    D \in \mathfrak{B}([0,1] \times [0,1]) = \mathfrak{B}([0,1]) \times \mathfrak{B}([0,1]) \subset \mathfrak{B}([0,1]) \times \mathcal{P}([0,1]).
\end{align*}
Therefore, $\chi_{D}$ is measurable. However,
\begin{align*}
    \int_{[0,1]} \left(\int_{[0,1]} \chi_D(x,y) \,d\lambda \right) \,d\mu & = \int_{[0,1]} \left(\int_{[0,1]} \chi_{\{x\}}(y) \,d\lambda \right) \,d\mu \\
    & = \int_{[0,1]} \lambda(\{x\}) \,d\mu = \int_{[0,1]} 1 \,d\mu \\
    & = \mu([0,1]) = 1,
\end{align*}
and
\begin{align*}
    \int_{[0,1]} \left(\int_{[0,1]} \chi_D(x,y) \,d\mu \right) \,d\lambda & = \int_{[0,1]} \left(\int_{[0,1]} \chi_{\{y\}}(x) \,d\mu \right) \,d\lambda \\
    & = \int_{[0,1]} \mu(\{y\}) \,d\lambda = \int_{[0,1]} 0 \,d\lambda = 0,
\end{align*}
which implies the Fubini theorem fails when the measure is not $\sigma$-finite.
\end{proof}


\section{Doubling and Hausdorff measures}



\medskip

\section{Differentiation}
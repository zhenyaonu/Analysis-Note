Let $A \subset X$ be a subset of $X$, and we want to assign {\em measure} $\mu(A)$ to this set. Denote by $\MM \subset P(X)$ the class of sets for which the measure $\mu$ is defined, where $P(X)$ is the power set of $X$, that is the set of all subsets of $X$. Elements of $\MM$ are called {\em measurable sets}. What properties do we need to define measure?
\begin{enumerate}[label=(\alph*)]
    \item $X \subset \MM$;
    
    \item If $A \in \MM$, then $X \setminus A \in \MM$;
    
    \item If $A_1, A_2, \cdot \in \MM$, then $\bigcup^\infty_{i=1}A_i \in \MM$.
\end{enumerate}

\medskip

\section{$\sigma$-algebra}

\begin{definition}\label{definition_sigma_algebra}
Let $X$ be a set. A collection $\MM$ of subsets of $X$ is $\sigma$-algebra if $\MM$ has the following properties:
\begin{enumerate}[label=(\alph*)]
    \item $X \in \MM$;
    
    \item If $A \in \MM$, then $X \setminus A \in \MM$;
    
    \item If $A_1, A_2, \cdots \in \MM$, then $\bigcup^\infty_{i=1}A_i \in \MM$.
\end{enumerate}
Then the $(X,\MM)$ is called measurable space and elements of $\MM$ are called measurable sets.
\end{definition}

\medskip

\begin{exercise}\label{exercise_11} {\rm \cite{1}}
Let $(X, \MM)$ be a measurable space. Show that
\begin{enumerate}[label=(\alph*)]
    \item $\emptyset \in \MM$;
    
    \item $A, B \in \MM$, then $A \setminus B \in \MM$;
    
    \item $\MM$ is closed under finite and countable unions and intersections.
\end{enumerate}
\end{exercise}
\begin{proof}
~\begin{enumerate}[label=(\alph*)]
    \item $\emptyset = X \setminus X$, then $\emptyset \in \MM$.
   
    \item By Definition \ref{definition_sigma_algebra}(c), $A \cup B \cup \emptyset \cup \emptyset \cup \cdots \in \MM$, which implies $A \cup B \in \MM$. 
    
    By De Morgan's Laws, we have 
    \begin{align*}
        X \setminus (A \cap B) = (X \setminus A) \cup (X \setminus B),
    \end{align*}
    and since $X \setminus A, X \setminus B \in \MM$, $A \cap B \in \MM$.
    
    Now, $A \setminus B = A \cap (X \setminus B)$, then by the results above, we have $A \setminus B \in \MM$.
    
    \item Suppose $A_1, A_2, \cdots \in \MM$, by De Morgan's Laws, we have
    \begin{align*}
        X \setminus \bigcap^\infty_{i=1} A_i = \bigcup^\infty_{i=1} (X \setminus A_i),
    \end{align*}
    and since $X \setminus A_i \in \MM$, then $X \setminus \bigcap^\infty_{i=1} A_i \in \MM$, and thus $\bigcap^\infty_{i=1} A_i \in \MM$. 
\end{enumerate}
\end{proof}

\medskip

\begin{example}[Examples of $\sigma$-algebra]
~\begin{enumerate}[label=(\alph*)]
    \item $P(X)$, or $2^X$, the family of all subsets of $X$ is a $\sigma$-algebra.
    
    \item $\MM = \{\emptyset, X\}$ is a $\sigma$-algebra.
    
    \item If $E \subset X$ is a fixed set, then $\MM = \{\emptyset, X, E, X\setminus E\}$ is a $\sigma$-algebra.
\end{enumerate}
\end{example}

\medskip

\begin{proposition}\label{prop_11}
A family $\MM$ of subsets of $X$ is a $\sigma$-algebra if and only if
\begin{enumerate}[label=(\alph*)]
    \item $X \in \MM$;
    
    \item If $A,B \in \MM$, then $A \setminus B \in \MM$;
    
    \item If $A_1, A_2, \cdots \in \MM$ are pairwise disjoint, then $\bigcup^\infty_{i=1} A_i \in \MM$.
\end{enumerate}
\end{proposition}
\begin{proof} 
~\begin{enumerate}
    \item[($\Rightarrow$)] (a) and (c) are obvious, it remains to show (b). By definition of $\sigma$-algebra, $A \cup B \cup \emptyset \cup \emptyset \cup \cdots \in \MM$, which implies $A \cup B \in \MM$. Also, by De Morgan's Laws, we have 
    \begin{align*}
        X \setminus (A \cap B) = (X \setminus A) \cup (X \setminus B),
    \end{align*}
    and since $X \setminus A, X \setminus B \in \MM$, $A \cap B \in \MM$. Now, $A \setminus B = A \cap (X \setminus B)$, then by the results above, we have $A \setminus B \in \MM$.
    
    \item[($\Leftarrow$)] It suffices to show that for any $A_1, A_2,\cdots \in \MM$, $\bigcup^\infty_{i=1}A_i \in \MM$. Let $B_1 = A_1$, $B_2 = A_2 \setminus A_1$, $B_3 = A_3 \setminus (B_1 \cup B_2)$ and so on. It gives a pairwise disjoint sequence of sets $B_1, B_2, \cdots \in \MM$. By the assumption, $\bigcup^\infty_{i=1}A_i = \bigcup^\infty_{i=1}B_i \in \MM$.
\end{enumerate}
\end{proof}

\medskip

\begin{proposition}
If $\{\MM_i\}_{i\in I}$ is a family of $\sigma$-algebras, then
\begin{align*}
    \MM = \bigcap_{i \in I} \MM_i
\end{align*}
is a $\sigma$-algebra. Note that $I$ can be uncountable.
\end{proposition}
\begin{proof}
~\begin{enumerate}[label=(\alph*)]
    \item For any $i \in I$, $\emptyset \in \MM_i$, then $\emptyset \in \bigcap_{i \in I} \MM_i$.
    
    \item If $A_i \in \bigcap_{i \in I} \MM_i$, then $A_i \in \MM_i$ for all $i \in I$, which implies $X\setminus A_i \in \MM_i$ for all $i \in I$. Hence, $X\setminus A_i \in \bigcap_{i \in I} \MM_i$.
    
    \item For $A_j \in \bigcap_{i \in I} \MM_i$, $j = 1, 2, \cdots$, it means that for all $j = 1, 2, \cdots$ and all $i \in I$, $A_j \in \MM_i$, and then for all $i \in I$, $\bigcup^\infty_{j=1} A_j \in \MM_i$. Thus, $\bigcup^\infty_{j=1} A_j \in \bigcap_{i \in I} \MM_i$.
\end{enumerate}

\end{proof}


\medskip

Let $\mathcal{R}$ be a family of subsets of $X$. Denote by $\sigma(\mathcal{R})$ the intersection of all $\sigma$-algebras that contain $\mathcal{R}$. Note that there is at least one $\sigma$-algebra that contains $\mathcal{R}$, namely the power set $P(X)$. We say that $\sigma(\mathcal{R})$ a {\em $\sigma$-algebra generated by $\mathcal{R}$}. Also, $\sigma(\mathcal{R})$ is the {\em smallest} $\sigma$-algebra that contains $\mathcal{R}$ in the sense that if $\MM$ is a $\sigma$-algebra that contains $\mathcal{R}$, then $\sigma(\mathcal{R}) \subset \MM$.

\medskip

\begin{example}
If $\mathcal{R} = \{E\}$, then $\sigma(\mathcal{R}) = \{\emptyset, X, E, X\setminus E\}$.
\end{example}

\medskip

\begin{theorem}{\rm \cite{2}}
If $\mathcal{R}$ is any collection of subsets of $X$, there exists a smallest $\sigma$-algebra $\sigma(\mathcal{R})$ in $X$ such that $\mathcal{R} \in \sigma(\mathcal{R})$.
\end{theorem}
\begin{proof}
Let $\Omega$ be the family of all $\sigma$-algebras $\MM$ in $X$ that contains $\mathcal{R}$. Since
the collection of all subsets of $X$ is such a $\sigma$-algebra, then $\Omega$ is not empty. Let $\MM^*$ be the intersection of all $\MM \in \Omega$. It is clear that $\mathcal{R} \subset \MM^*$ and that $\MM^*$ lies in every $\sigma$-algebra in $X$ which contains $\mathcal{R}$. It suffices to show that $\MM^*$ is itself a $\sigma$-algebra.

If $A_n \in \MM^*$ for $n = 1,2,3,\cdots$, and it $\MM \in \Omega$, then $A_n \in \MM$, so $\bigcup A_n \in \MM$, since $\MM$ is a $\sigma$-algebra. Since $\bigcup A_n \in \MM$ for every $\MM \in \Omega$, we conclude that $\bigcup A_n \in \MM^*$.
\end{proof}

\medskip
